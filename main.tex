\documentclass[openany]{report}
\usepackage[utf8]{inputenc}

\usepackage{stylesheets}
\usepackage{lecture_notes_styles}
\usepackage{pgfplots}
\pgfplotsset{compat=1.18}

\newcommand{\powerset}[0]{\mathcal{P}}

\title{MAT 3172 Lecture Notes}
\author{Last Updated:}

\begin{document}

\maketitle

\tableofcontents

%\chapter{Review}
\chapter{Probability Measures}
\section{Review of Set Theory}
Let $\Omega$ be an abstract set representing the sample space of a random experiment. The power set of $\Omega$ by $\powerset(\Omega)$ is defined to be the set of all subsets of $\Omega$. Elements of $\Omega$ are outcomes and its subsets are events. Therefore,
\[\powerset(\Omega) = \{A: A \subseteq \Omega\}\]
For $A,B \in \powerset(\Omega)$, we define
\[A \cup B = \{x: x \in A \text{ or } x \in B\}\]
\[A \cap B = \{x: x \in A \text { and } x \in B\}\]
\[\bar{A} = A^c = \{x: x \not\in A\}\]
\[A \Delta B = (A \cup B) \setminus (A \cap B)\]
In terms of events $A \cup B$ occurs if and only if at least one of the two events $A$ and $B$ occurs. Also, $A \cap B$ occurs if both $A$ and $B$ occurs. The empty set is denoted by $\emptyset$.\\[3ex]
\textbf{Examples of Sample Spaces:} When flipping a coin, we have two outcomes, so 
\[\Omega = \{H,T\}\]
If we flip a coin and role a dice, 
\[\Omega = \{1H, 2H, \ldots, 6H, 1T, \ldots 6T\}\]
If we flip a coin until we observe a head, 
\[\Omega = \{H, TH, TTH, TTTH, \ldots\}\]
Here, there are infinite outcomes so the sample space is infinite, but it is countable since we can list all the possibilities.\\[2ex]
If we pick a choose are sample space to be the points with distance one from the origin, we have the points in the unit circle,
\[
    \begin{tikzpicture}
        \begin{axis}[
            xmin=-2, xmax=2,
            ymin=-2, ymax=2,
            axis equal,
            axis lines=middle,
            xlabel={$x$},
            ylabel={$y$},
            samples=100 
        ]
        \addplot+[domain=0:360, no marks, thick] ({cos(x)}, {sin(x)});
        \end{axis}
    \end{tikzpicture}    
\]
The sample space is defined by 
\[\Omega = \{(x,y): d((x,y),(0,0)) \leq 1\} = \{(x,y): x^2 + y^2 \leq 1\} \]
In this example, the sample space omega is infinite as well, but it is uncountable.\\[2ex]
\textbf{Examples of Events:} An event is a subset of the sample space. For example, in the case of rolling a dice and flipping a coin, we have 
\[\Omega = \{1H, 2H, \ldots, 6H, 1T, \ldots 6T\}\]
And we can define an event $E$ as 
\[E = \{\text{Coin is heads and the dice is even}\} = \{2H, 4H, 6H\} \subset \Omega\]
If we flip a coin until the first head appears, 
\[\Omega = \{H, TH, TTH, \ldots\}\]
And we can define an event $E$ as
\[E = \{\text{First head appears before the 5th trial}\} = \{H, TH, TTH, TTTH, TTTTH\} \subset \Omega\]
\textbf{Examples of Power Sets:} Consider the sample space obtained by rolling a dice,
\[\Omega = \{1,2,3,4,5,6\}\]
And let $E = \{2,4,6\}$ be the event we roll an even number. Then, the power set of $E$ is 
\[\powerset(E) = \{\emptyset, \{1\}, \{2\}, \{3\}, \{1, 2\}, \{1, 3\}, \{2,3\}, \{1,2,3\}\}\]
The cardinality of the power set is 
\[|\powerset(E)| = 2^{|E|} = 8\]
\noindent
\textbf{Examples of Set Operations:}
\[\Omega = \{1,2,\ldots, 6\}\]
\[A = \{1,2,3\} \ B = \{1,2, 3\}\]
\[A \cup B = \{1,2,3,4,6\}\]
\[A \cap B = \{2\}\]
\[A^c = \{1,3,5\}\]
\[A \setminus B = \{x: x \in A, x \not\in B\} = \{4,6\}\]
\[A \Delta B = (A \cup B) \setminus (A \cap B)\]
\textbf{Example of Empty Set:}
Is $\{\{\}\} = \{\}$? \textbf{No}. $\{\{\}\}$ is a set with one element, which is the empty set, so $\{\{\}\} = \{\emptyset\}$.\\
\subsection{Properties of Sets}
\begin{itemize}
    \item $A \subset A$, $\emptyset \subset A$
    \item $A \subset B$ and $B \subset A$ implies $A = B$
    \item $A \subset C$ and $B \subset C$ implies $A \cup B \subset C$ and $A \cap B \subset C$. 
    \item $A \subset B$ if and only if $B^c \subset A^c$
    \item $(A^c)^c = A$, $\empty^c = \Omega$, $\Omega^c = \emptyset$
    \item $A \cup B = B \cup A$, $A \cap B = B \cap A$
    \item $A \cup A = A$, $A \cap \Omega = A$, $A \cup A^c = \Omega$, $A \cap A^c = \emptyset$. 
    \item $A \cap (B \cup C) = (A \cap B) \cup (A \cap C) \cup (A \cap C)$
    \item $A \cup (B \cap C) = (A \cup B) \cap (A \cup C)$
    \item $(A \cup B)^c = A^c \cap B^c$, $(A \cap B)^c = A^c \cup B^c$
\end{itemize}
\textbf{Example.} We have 
\[\bigcup_{n=1}^\infty \left[0, \frac{n}{n+1}\right) = [0,1)\]
To show that these sets are equal, consider the limit of the sequence $\left(\frac{n}{n+1}\right)_{n=1}^\infty$. As $n$ becomes large, the limit approaches 1. So, the union of all these sets will contain elements that become arbitrarly close to 1 but do not reach 1, so we have $[0,1)$.
\[\bigcap_{n=1}^\infty \left(0, \frac{1}{n}\right) = \emptyset\]
To show that these sets are equal, consider the sequence $\left(\frac{1}{n}\right)_{n=1}^\infty$, As this sequence approaches 0, the intersection of all these sets will contain elements that become arbitrarly close to 0 but do not reach 0, so we have the set $(0, 0)$ which is empty. Therefore, when taking the intersection of all these sents with $\left(0, \frac{1}{n}\right)$ which become arbitrarly small, we have the empty set.\\[2ex]
\noindent
\textbf{Example.} Prove that $A \Delta B = A^c \Delta B^c$.
\begin{proof}
    Note that $A \setminus B = A \cap B^c$. 
    \begin{align*}
        A^c \Delta B^c &= (A^c \cup B^c) \setminus (A^c \cap B^c)\\
        &= (A \cap B)^c \cap ((A \cup B)^c)^c\\ 
        &= (A \cap B)^c \cap (A \cup B) = (A \cup B) \setminus (A \cap B) \\
        &= A \Delta B  
    \end{align*}
\end{proof}
\section{Indicator Function}
Let $A \subset \Omega$. The indicator function of $A$ is defined as
\[I(x \in A) = I_A(x) = \begin{cases}
    1 & x \in A\\
    0 & x \not\in A
\end{cases}\]
\textbf{Example.} $A = [1,3]$
\[I_A(x) = I_{[1,3]} (x)\]
\[
\begin{tikzpicture}
    \begin{axis}[
        xmin=0, xmax=4,
        ymin=-0.2, ymax=1.2,
        axis lines=middle,
        xticklabels={1,2,3},
        yticklabels={0,1},
        xlabel={$x$},
        ylabel={$I_A(x)$},
        samples=2 
        ]
    \addplot+[thick, jump mark left, mark=none, color=blue] coordinates {
        (0,0)
        (0.99,0)
        (1,1)
        (3,1)
        (3.01,0)
        (4,0)
    };
    \end{axis}
    \end{tikzpicture}
\]
\subsection{Properties of Indicator Functions:}
\begin{itemize}
    \item $I_{A \cup B} = \max(I_A, I_B)$
    \item $I_{A \cap B} = I_A \cdot I_B$
    \item $I_{A \Delta B} = I_A + I_B \pmod{2}$
    \item $A \subset B$ if and only if $I_A \leq I_B$
    \item $I_{\cup_i A_i} \leq \sum_i I_{A_i}$
\end{itemize}

\section{Set Theoretic Limits}
\textbf{Exercise:} Prove that
\[I_{\cup_{i=1}^\infty A_i} = 1 - \prod_{i=1}^\infty(1 - I_{A_i})\]
and 
\[I_{A \Delta B}= (I_A - I_B)^2\]
\begin{definition}
    Let $\{A_n\}$ be a sequence of events. Then 
    \[\liminf A_n = \bigcup_{n=1}^\infty \bigcap_{m=n}^\infty A_m\]
    and
    \[\limsup A_n = \bigcap_{n=1}^\infty \bigcup_{m=n}^\infty A_m\]
\end{definition}
\noindent
\textbf{Example.} Suppose we have a sequence of events $A_1, A_2, A_3, \ldots$, we can define 
\[B_n = \bigcap_{m=n}^\infty A_m\]
So its sequence members are 
\[B_n = A_n \cap A_{n+1} \cap \cdots\]
\[B_{n+1} = A_{n+1} \cap A_{n+2} \cap \cdots\]
\[B_{n+2} = A_{n+2} \cap A_{n+3} \cap \cdots\]
These sets are getting smaller because surely the intersection of more sets will be smaller since $|A \cap B| \leq \min(|A|, |B|)$. So as we take more intersections, the sets become smaller and smaller. Now we can look at the union of these sets, 
\[\bigcup_{n=1}^\infty B_n = \liminf A_n\]
If instead we take $B_n$ to be the union of all the sets, 
\[B_n = \bigcup_{m=n}^\infty A_m\]
So the sequence members are 
\[B_n = A_n \cup A_{n+1} \cup \cdots\]
\[B_{n+1} = A_{n+1} \cup A_{n+2} \cup \cdots\]
\[B_{n+2} = A_{n+2} \cup A_{n+3} \cup \cdots\]
These sets are getting larger since we are taking the union of more sets, then we can look at the intersection of these sets,
\[\bigcap_{m=n}^\infty = \bigcap_{n=1}^\infty \bigcup_{m=n}^\infty A_n = \limsup A_n\]
\begin{lemma}
    We have 
    \[\limsup A_n = \left\{\omega: \sum_{i=1}^\infty I_{A_i}(\omega) = \infty \right\}\]
    and 
    \[\liminf A_n = \left\{\omega: \sum_{i=1}^\infty I_{A_i^c}(\omega) < \infty \right\}\]
    We can also express this as 
    \[\limsup_{n\rightarrow\infty} A_n = \left\{x \in X : \limsup_{n\rightarrow \infty} I_{A_n}(x) = 1\right\}\]
    and 
    \[\liminf_{n\rightarrow\infty} A_n = \left\{x \in X : \liminf_{n\rightarrow \infty} I_{A_n}(x) = 1\right\}\]
\end{lemma}

\begin{proof}
    If $\omega \in \limsup A_n$, then $\omega \in \bigcup\limits_{m=n}^\infty A_m$ for all integers $n$. Therefore, for any integer $n$ there exists an integer $k_n$ such that $\omega \in A_{k_n}$, since 
    \[\sum_{i=1}^\infty A_{A_i} (\omega) \geq \sum_{i=1}^\infty I_{A_{k_i}}(\omega) = \infty\]
    Conversely, for any integer $n$, by definition of the limit superior, 
    \[\sum_{i=n}^\infty I_{A_i}(\omega) = \infty\]
    This implies that $\omega \in \bigcup\limits_{j=n}^\infty A_j$ for all integers $n$. Therefore, 
    \[\omega \in \bigcap_{n=1}^\infty \bigcup_{m=n}^\infty A_m = \limsup A_n\]
    Then, we can notice that 
    \[\omega \in \liminf A_n = \bigcup_{n=1}^\infty \bigcap_{m=n}^\infty A_m \]
    implies that there exists an integer $n_0$ such that 
    \[\omega \in \bigcap_{k=n_0}^\infty A_k\]
    Therefore, 
    \[\sum_{n=1}^\infty I_{A_n^c} (\omega) = \sum_{n=1}^{n_0-1} I_{A_n^c} (\omega) \leq n_0 < \infty\]

\end{proof}
\textbf{Note:} For this reason, sometimes we write $\limsup A_n = A_n$ infinitely often. If $\liminf A_n = \limsup A_n$, then 
\[\lim A_n = \liminf A_n = \limsup A_n\]
\textbf{Remark:} The proof of the lemma above can be simplified by noticing the fact that 
\[(\limsup A_n)^c = \bigcup_{n=1}^\infty \bigcap_{m=n}^\infty A_m^c = \liminf A_n^c\]
\[(\liminf A_n)^c = \bigcap_{n=1}^\infty \bigcup_{m=n}^\infty A_m^c = \limsup A_n^c\]
To summarize, with the limit superior we have infinitely many cases where $I_{A_i}(\omega) = 1$. So what it means for $\omega \in \limsup A_n$ is that $\omega$ is in infinitely many of the $A_i$'s. For the limit inferior, it means that $\omega$ is in all but finitely many of the $A_i$'s.\\[2ex]
\noindent
\textbf{Example.}
Consider $\omega \in A_i$ when $i$ is odd, so 
\[\omega \in A_1, \omega \not\in A_2, \omega \in A_3, \omega \not\in A_4, \omega \in A_5, \ldots\]
$\omega$ is in infinitely many (but countable) number of the $A_i$'s. So $\omega \in \limsup A_n$. Now consider $\omega \in A_i$ when $i \geq 10$, so 
\[\omega \not\in A_1, \omega \not\in A_2, \omega \not\in A_3, \ldots, \omega \not\in A_9, \omega \in A_{10}, \omega \in A_{11}, \omega \in A_{12}, \ldots\]
So, $\omega$ is not in finitely many of the $A_i$'s, therefore $\omega \in \liminf A_n$.\\[2ex]
\noindent
\textbf{Example.} Consider sample space of flipping a coin and infinite number of times, and the event $E = \{HTTHT\}$, so the event that we get $HTTHT$ in that order. Because this outcome is possible, it will occur an infinite number of times in the sequence of events, so $E$ is in the limit superior, and the probability that $E$ occurs infinitely often is 1.
\begin{lemma}
    Let $\{A_n\}$ be a sequence of events, then 
    \begin{enumerate}
        \item If $A_n \subset A_{n+1}$ for any integer $n$, then 
        \[\lim A_n = \bigcup_{n=1}^\infty A_n\]
        \item If $A_{n+1} \subset A_n$ for any integer $n$, then
        \[\lim A_n = \bigcap_{n=1}^\infty A_n\]
    \end{enumerate}
\end{lemma}
\begin{proof}
    We can prove (1) and (2) similarly as follows, not that in this case, 
    \[\bigcap_{m=n}^\infty A_m = \bigcap_{m=1}^\infty A_m\]
    for all integers n. If $A_n \subset A_{n+1}$ for any integer $n$, then we have that 
    \[A_1 \subset A_2 \subset A_3 \subset \cdots \]
    So the set is getting bigger, now consider the limit superior 
    \[\limsup_{n\rightarrow\infty} A_n = \bigcap_{n=1}^\infty \bigcup_{m=n}^\infty A_m= \left(\bigcup_{m=1}^\infty A_m\right) \cap \left(\bigcup_{m=2}^\infty A_m\right) \cap \left(\bigcup_{m=3}^\infty A_m\right) \cap \cdots\]
    These sets are equal in size since if $A_1 \subset A_2$, then $A_1 \cup A_2 = A_2$. Therefore, we get that the intersection of these sets is $\bigcup\limits_{m=1}^\infty A_m$, and thus 
    \[\limsup_{n\rightarrow \infty} A_n = \bigcap_{n=1}^\infty\bigcup_{m=n}^\infty A_m = \bigcup_{n=1}^\infty A_n\]
    Furthermore, 
    \begin{align*}
        \liminf_{n\rightarrow\infty} A_n &= \bigcap_{n=1}^\infty \bigcup_{m=n}^\infty A_m\\
        &= \left(\bigcap_{m=1}^\infty A_m\right) \cup \left(\bigcap_{m=2}^\infty A_m\right) \cup \left(\bigcap_{m=3}^\infty A_m\right) \cup \cdots \\
        &=  A_1 \cup A_2 \cup A_3 \cup \cdots \\
        &= \bigcup_{n=1}^\infty A_n
    \end{align*}
    Therefore 
    \[\limsup A_n = \liminf A_n \implies \limsup A_n = \liminf A_n = \lim A_n\]
    The proof for (2) follows the same. 
\end{proof}
\noindent
\textbf{Example.} 
\[\lim_{n\rightarrow\infty} \left[0, 1 - \frac{1}{n}\right] = \lim_{n\rightarrow\infty} \left[0, 1 - \frac{1}{n}\right) = [0,1)\]
To see this, we have that 
\[A_1 = \{0\}, A_2 = \left[0,\frac{1}{2}\right], A_3 = \left[0,\frac{2}{3}\right],A_4 = \left[0,\frac{3}{4}\right], \ldots \]
So the set $A_n$ is increasing, therefore 
\[\limsup A_n = \liminf A_n = \lim A_n = \bigcup_{n=1}^\infty A_n = [0,1)\]
\textbf{Example.}
\[\lim_{n\rightarrow\infty} \left[0, 1 + \frac{1}{n}\right] = \lim_{n\rightarrow\infty} \left[0, 1 + \frac{1}{n}\right) = [0,1]\]
Similarly,
\[A_1 = [0,2], A_2 = \left[0,1 + \frac{1}{2}\right], A_3 = \left[0,1 + \frac{1}{3}\right],A_4 = \left[0,1 + \frac{1}{4}\right], \ldots \]
The set $A_n$ is decreasing, therefore
\[\limsup A_n = \liminf A_n =\bigcap_{n=1}^\infty A_n = \bigcap_{n=1}^\infty \left[0, 1 + \frac{1}{n}\right] = [0,1]\]
\noindent
\textbf{Example.} Let $B, C \subset \Omega$ and define the sequence 
\[A_n = \begin{cases}
    B & \text{if } n \text{ is odd}\\
    C & \text{if } n \text{ is even}
\end{cases}\]
Then we have 
\[\bigcup_{m=n}^\infty = B \cup C \implies \bigcap_{n=1}^\infty \bigcup_{m=n}^\infty A_m = B \cup C = \limsup A_n\]
Similarly for the limit inferior, 
\[\bigcap_{m=n}^\infty A_m = B \cap C \implies \bigcup_{n=1}^\infty \bigcup_{m=n}^\infty A_m = B \cup C = \liminf A_n\]
Therefore, we have 
\[\bigcap_{n=1}^\infty \bigcup_{m=n}^\infty A_m = B \cup C \text{ and } \bigcup_{n=1}^\infty \bigcap_{m=n}^\infty A_m = B \cap C\]
If $B \cap C \neq B \cup C$, then $B \cap C = \liminf A_n \neq \limsup A_n = B \cup C$. 
\section{Fields and Algebras}
\begin{definition}[Fields (Algebras)]
    A \emph{field} (or \emph{algebra}) is a class of subsets of $\Omega$ (called events) that contain $\Omega$ and are closed under finite union, finite intersection, and complementation. In otherwords, a family of subsets of $\Omega$ (say $\mathcal{A}$) is a field if 
    \begin{itemize}
        \item $\Omega \in \mathcal{A}$
        \item If $A \in \mathcal{A}$, then $A^c \in \mathcal{A}$
        \item If $A,B \in \mathcal{A}$, then $A \cup B \in \mathcal{A}$
    \end{itemize}
\end{definition}
\noindent
\textbf{Remarks.} If $A,B \in \mathcal{A}$ then $A \cap B \in \mathcal{A}$. This is true because 
\[(A^c \cup B^c)^c = A \cap B\]
\begin{definition}[$\sigma$-field]
    A $\sigma$-field (or $\sigma$-algebra) is a field that is closed under countable union (which implies that it is closed under countable intersection).
\end{definition}
\noindent
\textbf{Example.} Let $\Omega$ be a set and $A,B \subset \Omega$. Then, 
\[\mathcal{A} = \{\Omega, \emptyset, A, A^c, B, B^c, A\cup B, A \cap B, A \cap B^c, A^c \cap B, A^c \cup B^c, A^c \cap B^c\}\]
\textbf{Examples of $\sigma$-fields.}
\begin{itemize}
    \item The power set $\mathcal{P}(\Omega)$
    \item $\mathcal{F} = \{\Omega, \emptyset\}$
    \item The family of subsets or $\real$ which are either countable or their complements are countable. 
    \item Let $\mathcal{B}$ be the smallest $\sigma$-field containing all open sets. Then $\mathcal{B}$ is called the Borel $\sigma$-field.
\end{itemize}


\begin{definition}[Probability Measure]
    Let $\Omega$ be a sample space and $\mathcal{F}$ be a $\sigma$-field on $\Omega$. A probability measure $P$ is defined on $\mathcal{F}$ such that 
    \begin{enumerate}[label=(\roman*)]
        \item $P(\Omega) = 1$
        \item If $A_1, A_2, \ldots \in \mathcal{F}$ are disjoint, then 
        \[P\left(\bigcup_{i=1}^\infty A_i\right) = \sum_{i=1}^\infty P(A_i)\]
    \end{enumerate}
\end{definition}
\subsection{Properties of Probability Measures}
\begin{enumerate}[label=(\roman*)]
    \item Since $P(\Omega) = 1 = P(\Omega \cup \emptyset) = P(\Omega) + P(\emptyset)$, we have $P(\emptyset) = 0$
    \item Since $(A\setminus B) \cup (A \cap B) = A$ and $(A\setminus B) \cap (A \cap B) = \emptyset$, we have
    \[P(A \setminus B) = P(A) - P(A \cap B)\]
    \item Similarly, $(A \setminus B) \cup B = A \cup B$ and $(A \setminus B) \cap B = \emptyset$, which implies 
    \[P(A \cup B) = P(A) + P(B) - P(A \cap B)\]
    \item If $A \subset B$ then $A \cup (B \setminus A) = B$. Therefore, 
    \[P(A) + P(B \setminus A) = P(B)\]
    and furthermore,
    $P(A) \leq P(B)$
\end{enumerate}
\subsection{Expectations}
\begin{definition}
    
    Let $X: \Omega \to \real$, an expecation $E$ be an operator with the following properties, 
    \begin{enumerate}[label=(\roman*)]
        \item If $X \geq 0$ then $E(X) \geq 0$
        \item If $c \in \real$ is a constant, then $E(cX) = cE(X)$
        \item $E(X_1 + X_2) = E(X_1) + E(X_2)$
        \item $E(1) = 1$
        \item If $X_n(\omega)$ is monotonically increasing and $X_n(\omega) \rightarrow X(\omega)$, then 
        \[\lim_{n\rightarrow \infty} E(X_n) = E(X)\]
    \end{enumerate}
\end{definition}
\noindent
\textbf{Example.} Flip 2 coins, 
\[\Omega = \{HH, HT, TH, TT\}\]
Define $X: \Omega \rightarrow \real$, with 
\[X(HH) = 2, X(HT) = X(TH) = 1, X(TT) = 0\]
So $X$ is a random variable which represents the number of heads.\\[2ex]
\textbf{Example.} Flip a coin until a head appears, 
\[\Omega = \{H, TH, TTH, TTTH, \ldots\}\]
With 
\[X(H) = 1, X(TH) = 2, X(TTH) = 3, X(TTTH) = 4, \ldots\]
So $X$ is a random variable which represents the number of trials until a head appears.\\[2ex]
\textbf{Example.} Define 
\[\Omega = \{(x,y): x^2 + y^2 \leq 1\}\]
and 
\[X(x,y) = \sqrt{x^2 + y^2} = \text{ Distance of $(x,y)$ from $(0,0)$}\]
\textbf{Example.} Let $X$ be a random variable, and
\[E(X) = \lim_{D \rightarrow \infty} \frac{\int_{-D}^D X(\omega)d\omega}{2D}\]
Check if $E$ satisfies the definition for an expectation. \\[2ex]
\textbf{Solution.} We have to check the 5 axioms, 
\begin{enumerate}
    \item Its clear that if $X \geq 0$ then $E(X) \geq 0$ since the integral of a non-negative function is non-negative.
    \item 
    \[E(cX) = \lim_{D \rightarrow \infty} \frac{\int_{-D}^D cX(\omega) d\omega}{2D} = c\lim_{D \rightarrow \infty}\frac{\int_{-D}^DX(\omega)d\omega}{2D} =  cE(x)\]
    \item 
    \begin{align*}
        E(X_1 + X_2) &= \lim_{D \rightarrow \infty} \frac{\int_{-D}^D (X_1(\omega) + X_2(\omega)) d\omega}{2D}\\
        &=\lim_{D \rightarrow \infty} \left(\frac{\int_{-D}^D X_1(\omega) d\omega}{2D} + \frac{\int_{-D}^D X_2(\omega) d\omega}{2D}\right)\\
        &=\lim_{D \rightarrow \infty} \frac{\int_{-D}^D X_1(\omega) d\omega}{2D} + \lim_{D \rightarrow \infty}\frac{\int_{-D}^D X_2(\omega) d\omega}{2D}\\
        &= E(X_1) + E(X_2)
    \end{align*}
    \item 
    \[E(1) = \lim_{D \rightarrow \infty} \frac{\int_{-D}^D 1 d \omega}{2D} = \lim_{D\rightarrow \infty} \frac{2D}{2D} = 1\]
    \item 
    The 5th axiom fails however. Take $\Omega = \mathbb{R}$ and $X_n (\omega) = I_{[-n,-n]}(\omega)$, then 
    \[\lim_{D \rightarrow \infty} \frac{\int_{-D}^D X(\omega)d\omega}{2D} = 0\]
    But, $x_n(\omega) \rightarrow 1$. So the operator in this example is not a proper form of expectation. 
\end{enumerate}
\section{Finding Probabilities Using Expectations}
\begin{definition}
    For any event $A$, define 
    \[P(A) = E(I_A(\omega))\]
    For simplicity, we somtimes drop $\omega$ and write 
    \[P(A) = E(I_A)\]
\end{definition}
\subsection{Properties}
\begin{enumerate}
    \item $E\left(\sum\limits_{i=1}^n c_iX_i\right) = \sum\limits_{i=1}^n c_iE(X_i)$
    \item If $X \leq Y \leq Z$, then $E(X) \leq E(Y) \leq E(Z)$
    \item If $\{A_i\}$ is a sequence of events, then 
    \[P\left(\bigcup_{i=1}^\infty A_i\right) \leq \sum_{i=1}^\infty P(A_i)\]
   
\end{enumerate}
The third property (known as Borel's inequality) is an important result and can be proved as follows. 
\begin{proof}
    Using the fact that 
    \[I_{\cup A_i} \leq \sum_{i=1}^\infty I_{A_i}\]
    and the second property where if $X \leq Y$, then $E(X) \leq E(Y)$, then 
    \[E\left(I_{\cup A_i}\right) \leq \sum_{i=1}^\infty E(I_{A_i})\]
    Then this is exactly 
    \[P\left(\bigcup_{i=1}^\infty A_i\right) \leq \sum_{i=1}^\infty P(A_i)\]
\end{proof}
\noindent
We can use these properties to prove the axioms of probability measures. 
\begin{enumerate}
    \item $P(\Omega) = E(I_\Omega)$, we have $I_\Omega(\omega) = 1$ $\forall \omega \in \Omega$, so $P(\Omega) = E(I_\Omega) = 1$
    \item We want to show that the probability of disjoint events is the sum of their probablities, so we have that 
    \[P\left(\bigcup_{i=1}^\infty A_i\right) = E\left(I_{\cup A_i}\right)\]
    We know from indicator functions that 
    \[I_{\cup A_i} = \sum_{i=1}^\infty I_{A_i}\]
    Therefore,
    \[E\left(I_{\cup A_i}\right) = E\left(\sum_{i=1}^\infty I_{A_i}\right) = \sum_{i=1}^\infty E(I_{A_i}) = \sum_{i=1}^\infty P(A_i)\]
    as required. 
\end{enumerate}
Another useful result we can prove is that 
\[P(A \cup B) = P(A) + P(B) - P(A \cap B)\]
 and furthermore 
 \[P(A \cup B \cup C) = P(A) + P(B) + P(C) - P(A \cap B) - P(A \cap C) - P(B \cap C) + P(A \cap B \cap C)\]
\begin{proof}
    We know that 
    \[I_{A \cup B \cup C} = 1 - (1 - I_A)(1 - I_B)(1 - I_C)\]
    This can be easily shown by considering the cases when we have $\omega \in A \cup B \cup C$, then at least one of the terms $(1 - I_A)$, $(1 - I_B)$, or $(1 - I_c)$ will be zero, then
    \[1 - (1 - I_A)(1 - I_B)(1 - I_C) = 1 - 0 = 1 = I_{A \cup B \cup C}\]
    And if $\omega \not\in A \cup B \cup C$, then $\omega \not\in A$ and $\omega \not\in B$ and $\omega \not\in C$, so all their indicator functions will be zero, then we get 
    \[1 - (1 - 0)(1 - 0)(1-0) = 0 = I_{A \cup B \cup C}\]
    So, we have the result that 
    \[I_{\cup A_i} = 1 - \prod_{i=1}^\infty (1 - A_i)\]
    Now if we expand the equality, we get 
    \begin{align*}
        I_{A \cup B \cup C} &= 1 - (1 - I_A)(1 - I_B)(1 - I_C) \\
        &= 1 - (1 - I_A - I_B + I_A I_B)(1 - I_C)\\
        &= 1 - 1 + I_A + I_B + I_C - I_A I_B - I_A I_C - I_B I_C + I_A I_B I_C\\ 
        &= I_A + I_B + I_C - I_A I_B - I_A I_C - I_B I_C + I_A I_B I_C
    \end{align*}
    Then, we can take the expected value of both sides, 
    \begin{align*}
        E(I_{A \cup B \cup C}) &= E(I_A + I_B + I_C - I_A I_B - I_A I_C - I_B I_C + I_A I_B I_C)\\
        &= E(I_A) + E(I_B) + E(I_C) - E(I_A I_B) - E(I_A I_C) - E(I_B I_C) + E(I_A I_B I_C)\\
        &= P(A) + P(B) + P(C) - P(A \cap B) - P(A \cap C) - P(B \cap C) + P(A \cap B \cap C)\\
        &= P(A \cup B \cup C)
    \end{align*}
    Therefore, 
    \[P(A \cup B \cup C) = P(A) + P(B) + P(C) - P(A \cap B) - P(A \cap C) - P(B \cap C) + P(A \cap B \cap C)\]
    Showing that $P(A \cup B) = P(A) + P(B) - P(A \cap B)$ follows the same.
\end{proof}
Then we get that the general case for the union of $n$ events as 
\[P\left(\bigcup_{i=1}^n A_i\right) = \sum_{i=1}^n P(A_i) -\sum_{1 \leq i < j \leq n}P(A_i \cap A_j) + \sum_{i \leq i < j < k \leq n} P(A_i \cap A_j \cap A_k) + (-1)^{n+1}P(A_1 \cap A_2 \cap \cdots \cap A_n)\]
And this is again proved using the fact that 
\[I_{\cup_{i=1}^n A_i} = 1 - \prod_{i=1}^n (1- A_i)\]
\textbf{Example.} (Confused Secretary Problem) Suppose we have 100 distinct letters to be sent to 100 different people. The Secretary confuses the addresses, and sends 100 letters at random to these 100 people. What is the probability that at least one letter is sent to the correct address?\\[2ex]
\textbf{Solution.} Let $A_i$ denote the event that the $i$th letter goes to the right person. We want to find the probability that one of these events occurs, so $P\left(\bigcup\limits_{i=1}^{100} A_i\right)$. From the formulas previously discussed, 
\begin{align*}
    P\left(\bigcup_{i=1}^{100} A_i\right) &= P(A_1) + \cdots P(A_100) - P(A_1 \cap A_2) - \cdots - P(A_{99} \cap A_{100})\\
    &+ P(A_1 \cap A_2 \cap A_3) + \cdots + P(A_{98} \cap A_{99} \cap A_{100})\\
    &+ P(A_1 \cap A_2 \cap A_3 \cap A_4) - \cdots - P(A_{97} \cap A_{98} \cap A_{99} \cap A_{100})\\
    &+ \cdots - P(A_1 \cap A_2 \cap \cdots \cap A_{100})
\end{align*}
Then we can evaluate each probability, 
\[P(A_1) = \frac{1}{100}, \ P(A_2) = \frac{1}{100}, \ldots, P(A_{100}) = \frac{1}{100}\]
\[P(A_1 \cap A_2) = \frac{1}{100} \cdot \frac{1}{99} = \cdots = P(A_{99} \cap A_{100})\]
\[P(A_1 \cap A_2 \cap A_3) = \frac{1}{100} \cdot \frac{1}{99} \cdot \frac{1}{98} = \cdots\]
Then continuing this sequence we have
\begin{align*}
    P\left(\bigcup_{i=1}^{100}\right) &= 100\left(\frac{1}{100}\right) - {100 \choose 2}\left(\frac{1}{100\cdot99}\right)+ \cdots - \cdots - {100 \choose 100}\left(\frac{1}{100 \cdot 99 \cdots 1}\right)\\
    &= 1 - \frac{100 \cdot 99}{2!} \cdot \frac{1}{100 \cdot 99} + \frac{100 \cdot 99 \cdot 98}{3!}\cdot\frac{1}{100 \cdot 99 \cdot 98} - \cdots - \frac{1}{100!}\\
    &= 1 - \frac{1}{2!} + \frac{1}{3!} - \frac{1}{4!} + \frac{1}{5!} - \cdots - \frac{1}{100!}\\ 
\end{align*}
Now recall that $e^x$ can be written as the infinite series 
\[e^x = \sum_{n=1}^\infty \frac{x^n}{n!}\]
So, 
\[e^{-1} = 1 - 1 + \frac{1}{2!} - \frac{1}{3!} + \frac{1}{4!} - \cdots\]
\[1 - e^{-1} = 1 - \frac{1}{2!} + \frac{1}{3!} - \frac{1}{4!} + \cdots\]
Now since $100!$ is incredibly large, it is sufficiently large for this infinite sum. So we get that 
\[P(\text{At least one letter goes to the right address}) = 1 - \frac{1}{e}\]
\begin{lemma}[Fatou's Lemma]
    If $\{A_n\}$ is a family of events, then 
    \begin{enumerate}
        \item $P(\liminf A_n) \leq \liminf P(A_n) \leq \limsup P(A_n) \leq P(\limsup A_n)$
        \item If $\lim A_n = A$, then $\lim P(A_n) = P(A)$
    \end{enumerate}
\end{lemma}
To prove this lemma, we first need to prove the following lemma. 
\begin{lemma}\label{lem:1.5.2}
    If $A_n \subset A_{n+1}$ for any $n \in \nat$, then $\lim P(A_n) = P(A)$. Similarly, if $A_{n+1} \subset A_n$ for any $n \in \nat$, then $\lim P(A_n) = P(A)$ where in both cases $\lim A_n = A$.
\end{lemma} 
\begin{proof}
    If $A_n$ is increasing (i.e $A_n \subset A_{n+1}$), start by defining $B_n = A_n \setminus A_{n-1}$ with $B_1 = A_1$. Then, 
    \[\bigcup_{i=1}^n B_i = \bigcup_{i=1}^n A_i= A_n\]
    The point of this is now we have that the $B_i$'s are disjoint. Now if we take the limit we get 
    \[\bigcup_{i=1}^\infty B_i = \bigcup_{i=1}^\infty A_i = A\]
    Taking the probability now, 
    \[P(A) = P\left(\bigcup_{i=1}^\infty B_i\right) = \sum_{i=1}^\infty P(B_i) = \lim_{n\rightarrow\infty} \sum_{i=1}^n P(B_i) = \lim_{n\rightarrow\infty}P(A_n)\]
    If $A_n$ is decreasing ($A_{n+1} \subset A_n$), then $A_n^c$ is increasing. So, 
    \[\lim_{n\rightarrow\infty}P(A_n^c) = \lim(1 - P(A_n)) = P(A^c) = 1 - P(A)\]
    Therefore, 
    \[\lim_{n\rightarrow\infty} P(A_n) = A\]
\end{proof}
\noindent
To summarize, when $A_n$ is increasing, we have 
\[\lim P(A_n) = P\left(\bigcup_{i=1}^\infty A_i\right) = P\left(\lim_{n\rightarrow\infty}A_n\right)\]
when $A_n$ is decreasing, 
\[\lim P(A_n) = P\left(\bigcap_{i=1}^\infty A_i\right) = P\left(\lim_{n\rightarrow\infty}A_n\right)\]
Now we can prove Fatou's Lemma.
\begin{proof}
    To prove (1), notice that from the first part of lemma 1.5.2, we can write 
    \[P(\liminf A_n) = P\left(\lim_{n\rightarrow\infty} \cap_{i=n}^\infty A_i\right) = \lim_{n\rightarrow \infty} P \left(\cap_{i=n}^\infty A_i\right)\leq \liminf P(A_n)\]
    since $\bigcap\limits_{i=n}^\infty A_i \subset A_n$. Likewise, 
    \[P(\limsup A_n) = P\left(\lim_{n\rightarrow\infty} \bigcup_{i=n}^\infty A_i\right) = \lim_{n\rightarrow\infty} P\left(\bigcup_{i=1}^\infty A_i\right) \geq \limsup P(A_n)\]
    since $A_n \subset \bigcup\limits_{i=n}^\infty A_i$. For part (2), notice that if 
    \[A = \limsup A_n = \liminf A_n\]
    we have 
    \[P(A) = P(\liminf A_n) \leq \liminf P(A_n) \leq \limsup P(A_n) \leq  P(\limsup A_n) = P(A)\]
    Then this implies that 
    \[\lim_{n\rightarrow\infty} P(A_n) = P(A)\]
\end{proof}


\section{Independence}
\begin{definition}
    Let $A$ and $B$ be events. We say that $A$ and $B$ are independent if 
    \[P(A \cap B) = P(A)P(B)\]
\end{definition}
\noindent
\textbf{Example.} Flip a coin and roll a die, 
\[\Omega = \{1H, 2H, \ldots, 6H, 1T, 2T, \ldots, 6T\}\]
Let $A = \{1H, 1T\}$ and $B = \{1H, 2H, 3H,4H, 5H,6H\}$. Are $A,B$ independent? 
\[A \cap B = \{1H\} \neq \emptyset\]
\[P(A \cap B) = \frac{1}{12}\]
\[P(A) = \frac{1}{6}, \ P(B) = \frac{1}{2}\]
\[P(A)P(B) = \frac{1}{6}\cdot \frac{1}{2} = \frac{1}{12} = P(A\cap B)\]
Therefore, $A$ and $B$ are independent. 
\subsection{Properties of Independence}
\begin{enumerate}
    \item  If $A$ and $B$ are independent, then $A$ and $B^c$ are independent, $A^c$ and $B$ are independent, and $A^c$ and $B^c$ are independent.
    \item If $A,B,C$ are indepdent then $A$ and $B \cup C$ are independent. similarly $A$ and $B \cap C$ are independent.
    \item An event $A$ is independent of itself if and only if $A = \emptyset$ or $A = \Omega$.
    \item Any event $A$ is independent of $\Omega$.
\end{enumerate}

\begin{proof}
    \begin{enumerate}
        \item To prove that $A^c$ and $B$ are independent, recall that
        \[P(A^c \cap B) = P(B) - P(A \cap B)\]
        So, 
        \begin{align*}
            P(A^c \cap B) &= P(B) - P(A \cap B)\\
            &= P(B) - P(A)P(B)\\
            &= P(B)(1 - P(A))\\
            &=P(B)P(A^c)
        \end{align*}
        The proof for $A$ and $B^c$ follows the same. To prove $A^c$ and $B^c$ are independent, 
        \begin{align*}
            P(A^c \cap B^c) &= P((A\cup B)^c)\\
            &= 1 - P(A\cup B)\\
            &= 1 - [P(A) + P(B) - P(A\cap B)]\\
            &= 1 - P(A) - P(B) + P(A)P(B)\\
            &= 1 - P(A) - P(B)(1 - P(A))\\
            &= (1-P(A))(1-P(B))\\
            &= P(A^c)P(B^c)
        \end{align*}
        \item Given 3 independent events $A,B,C$, then any operations between the sets is independent. So we can show $A$ and $B \cup C$ is independent since 
        \begin{align*}
            P(A \cap (B \cup C)) &= P((A \cap B) \cup (A \cap C)) \\
            &= P(A \cap B) + P(A \cap C) - P(A \cap B \cap A \cap C)\\
            &= P(A)P(B) + P(A)P(B) - P(A)P(B)P(C)\\
            &= P(A)(P(B) + P(C) - P(B)P(C))\\
            &= P(A)P(B \cup C)
        \end{align*}
        Similarly for $A$ and $B \cap C$, 
        \begin{align*}
            P(A \cap (B \cap C)) &= P((A \cap B) \cap (A \cap C)) \\
            &= P(A \cap B \cap A \cap C)\\
            &= P(A)P(B)P(C)\\
            &= P(A)P(B \cap C)
        \end{align*}
        \item ($\implies$) If $A$ is independent of itself, then 
        \[P(A \cap A) = P(A)P(A) \implies P(A) = P(A)^2 \implies P(A) =0, 1 \implies A = \emptyset, A = \Omega\]
        ($\impliedby$) If $A = \emptyset$, then 
        \[P(A \cap A) = P(A) = 0 = P(\emptyset)P(\emptyset)\]
        If $A = \Omega$, then
        \[P(A \cap A) = P(A) = 1 = P(\Omega)P(\Omega)\]
        \item Every event is independent of $\Omega$ since
        \[P(A \cap \Omega) = P(A)P(\Omega) = P(A) \cdot 1\]
    \end{enumerate}
    
\end{proof}
\noindent

\begin{lemma}[Borel Cantelli Lemma]
    Let $(\Omega, \mathcal{F}, P)$ be a probability space and let $\{E_i\}$ be a sequence of events. Then,
    \begin{enumerate}[label=(\roman*)]
        \item If $\sum\limits_{i=1}^\infty P(E_i) < \infty$, then $P(\limsup E_n )= 0$
        \item If $\{E_i\}$ is a sequence of independent events, then $P(\limsup E_n) = 0$ or 1 according to to whether the series $\sum\limits_{i=1}^\infty P(E_i)$ diverges or converges respectively.
    \end{enumerate}
\end{lemma}
\begin{proof}
    Set $E = \limsup E_n$. We have $E = \bigcap\limits_{n=1}^\infty F_n$ where $F_n = \bigcup\limits_{m=n}^\infty E_m$. For every positive integer $n$, 
    \[0 \leq P(F_n) \leq \sum_{m=n}^\infty P(E_m)\]
    Since $\sum\limits_{i=1}^\infty P(E_i) < \infty$, then $\lim\limits_{n\rightarrow \infty} P(E_n) = 0$. Since $F_n \downarrow E$, from Fatou's lemma we can write 
    \[0 = \lim_{n\rightarrow \infty} P(F_n) = P \left(\lim_{n\rightarrow\infty} F_n\right) = P(\limsup E_n)\]
    Thus (i) is proved. For (ii), suppose $E_1, E_2, \ldots$ are independent. From (ii), we know that $P(\limsup E_n) = 0$  if the sum $\sum_{n=1}^\infty E_n$ is finite. It remains to show that if the sum is infinite, then $E_n$ occurs infinitely often. Let $E = \limsup E_n$. Then 
    \[E^c = \liminf E_n^c\]
    The sequence of events $\{E_n^c\}$ are also independent, so we have 
    \[P\left(\bigcap_{m=n}^\infty E_m^c\right) \leq P\left(\bigcap_{m=n}^N E_m^c\right) = \prod_{m=n}^N (1 - P(E_m)) \leq \exp \left(- \sum_{m=n}^N P(E_m)\right)\]
    This inequality comes from the Talor Series expansion of $e^x$, so
    \[e^{-x} = 1 - x + \frac{x^2}{2!} - \frac{x^3}{3!} + \cdots\] 
    \[e^{-x} \geq 1 - x\]
    As $N \rightarrow \infty$, we get $\bigcap\limits_{m=n}^N E_{m}^C \downarrow E^c$. Thus, 
    \[P\left(\bigcap_{m=n}^N E_m^c\right) \leq \exp\left(-\sum_{m=n}^N P(E_m)\right) \leq \exp \left(-\sum_{m=n}^N P(E_m)\right) \rightarrow 0\]
    This implies that $P(E^c) = 0$, thus $P(E) = 1$.
 \end{proof}
 \begin{corollary}
    If $\{E_i\}$ is a sequence of independent events then 
    \[P(\limsup E_n) =  0 \iff \sum_{i=1}^{\infty} P(E_i) < \infty\]
\end{corollary}
\begin{proof}
    ($\impliedby$) We know from the Borel Cantelli lemma that if $\sum\limits_{i=1}^\infty P(E_i) < \infty$, then $P(\limsup E_n) = 0$.\\[1ex]
    ($\implies$) Suppose $P(\limsup E_n) = 0$. Then from part (2), $P(\limsup E_n) = 1$ when $\sum\limits_{i=1}^\infty P(E_i) = \infty$. So it must be that $\sum\limits_{i=1}^\infty P(E_i) < \infty$ as required. 
\end{proof}
\noindent
\textbf{Remark.} Notice that indepdence is required by Corollary 1.6.1. To see this, let $(\Omega = [0,1], B, P)$ be a probability space with 
\[P(A) = \int_A dx\]
for a Borel set $A$. It's easy to show that $P$ is a probability measure on $[0,1]$. Now define $E = \left(0, \frac{1}{n}\right)$ and notice that $E_n \downarrow \emptyset$. Therefore, 
\[P(\limsup E_n) = P(\lim E_n) = 0\]
Since 
\[P(E_n) = \int_0^\frac{1}{n} dx = \frac{1}{n}\]
we have 
\[\sum_{n=1}^\infty P(E_n) = \sum_{n=1}^\infty \frac{1}{n} = \infty\]
This does not violate Corollary 1.6.1 since the events $E_n$ are not independent. For example. consider $E_2 = (0,1/2)$, and $E_3 = (0,1/3)$. Then 
\[P(E_2) = \frac{1}{2}, \ \ P(E_3) = \frac{1}{3}\]
\[P(E_2 \cap E_3) = P(E_3) = \frac{1}{3} \neq P(E_2)P(E_3) = \frac{1}{6}\]
\textbf{Example.} Let $E_n$ denote the event that the result of a fair coin flip is heads on both the $n$th and the $(n + 1)$st toss. Then,
$\limsup E_n$ is the event that in repeated tossing of a fair coin two
successive head appears infinitely often times. Since ${E_{2n}}$ is an
independent sequence of events and
\[\sum_{n=1}^\infty P(E_{2n}) = \sum_{n=1}^\infty \frac{1}{4} = \infty\]
we have $P(\limsup E_{2n}) = 1$. This implies that $P(\limsup E_n) = 1$ since 
\[\limsup E_{2n} \subset \limsup E_n\]
\chapter{Important Inequalities and Probability Theorems}
In this chapter we review some important inequalities in
probability. We start with some definitions and review of probability topics. 
\begin{definition}[Probability Mass Density Function]
    A function $f: \real \rightarrow \real$ is called a probability density function (p.d.f) if
    \begin{enumerate}[label=(\roman*)]
        \item $f \geq 0$
        \item $\int f(x)dx = 1$
    \end{enumerate}
    similarly $f$ is a probability mass function defined on discrete random variables with 
    \begin{enumerate}
        \item $f \geq 0$
        \item $\sum f(x) = 1$
    \end{enumerate}
\end{definition}
\begin{definition}[Cumulative Density Function]
    The cumulative density function (c.d.f) of a random variable $X$ is defined as
    \[F(X) = P(X \leq x)\]
    In otherwords, its the integral (or sum if $X$ is discrete) of the p.d.f. 
\end{definition}
\begin{definition}
    A random variable (r.v.) is of continuous type if its cumulative distribution function (c.d.f) is continuous. 
\end{definition}

\begin{definition}
    We define the expected value of a random variable $X$ as
    \[E(X) = \mu = \int_{-\infty}^\infty xdF(x)\]
    where $dF(x) = fdx$ is the probability density function (p.d.f) of $X$. The variance is given as 
    \[\Var(X) = \sigma^2 = E[(X - \mu)^2] = \int_{-\infty}^\infty (x - \mu)^2dF(X) = E(X^2) - E(X)^2\]
\end{definition}
\noindent
\textbf{Example.} Let $X$ be ae random variable with p.d.f 
\[f(x) = \begin{cases}
    \frac{1}{3} & \text{if } 0 \leq x < 0.5\\
    \frac{5}{3} & \text{if } 0.5 \leq x \leq 1\\
\end{cases}\]
\begin{enumerate}[label=(\roman*)]
    \item Calculate $F(x)$. Is $X$ a continuous random variable?
    \item Find $E(X)$ and $\Var(X)$. 
    \item Find $P(0.5 < X < 0.75)$.
\end{enumerate}
\textbf{Solution.}
\begin{enumerate}[label=(\roman*)]
    \item To find the c.d.f, we integrate the p.d.f to get 
    \[F(X) = \begin{cases}
        0 & \text{if } x < 0\\
        \int_0^x \frac{1}{3}dt =\frac{x}{3} & \text{if } 0 \leq x < 0.5\\
        \int_{0}^{0.5} \frac{1}{6}dt + \int_{0.5}^x \frac{5}{3}dt = \frac{5x}{3} - \frac{2}{3} & \text{if } 0.5 \leq x \leq 1\\
        1 & \text{if } x > 1
    \end{cases}\]
    $X$ is a continuous random variable since its c.d.f is continuous. 
    \item To find $E(X)$, we use the definition of expected value,
    \[E(X) = \int_0^1 xf(x)dx = \int_0^{0.5} \frac{x}{6}dx + \int_{0.5}^1 \frac{5x}{3}dx = \frac{2}{3}\]
    \[E(X) = \int_0^1 x^2f(x)dx = \int_0^{0.5} \frac{x^2}{6}dx + \int_{0.5}^1 \frac{5x}{3}dx = \frac{1}{2}\]
    Therefore $E(X) = \frac{2}{3}$ and $\Var(X) = \frac{2}{3} - \frac{4}{9} = \frac{1}{16}$
    \item
    \begin{align*}
        P(0.5 < X < 0.75) &= P(X < 0.75) - P(X < 0.5)\\
        &= F(0.75) - F(0.5)\\
         &= \int_{0.5}^{0.75} f(x)dx\\
         &= \int_{0.5}^{0.75} \frac{5}{3}dx\\
         &= \frac{5}{12}  
    \end{align*}
    All the above ways of calculating $P(0.5 < X < 0.75)$ are equivalent.
\end{enumerate}
\section{Important Inequalities}
\begin{theorem}[Markov's inequality]
    Let $X \geq 0$ be a random variable and $a$ be a positive constant. Then 
    \[P(X \geq a) \leq \frac{E(X)}{a}\]
\end{theorem}
\begin{proof}
    To prove this, notice that 
    \[I(X \geq a) \leq \frac{X}{a}\]
    This is because for a finite $a$, we have when $X < a$,
    \[\frac{X}{a} \geq 0 = I(X \geq a)\]
    Then when $X \geq a$,
    \[\frac{X}{a} \geq 1 = I(X \geq a)\]
    so in either case $\frac{X}{a} \geq I(X \geq a)$. Then we can take expected value of each side of the inequality and get 
    \[E(I(X \geq a)) = P(X \geq a) \leq E\left(\frac{X}{a}\right) = \frac{E(X)}{a} \] 
\end{proof}
Similarly, we can write 
\[I(|X-b|\geq a) \leq \frac{(X-b)^2}{a^2}\]
for all $a > 0$ and $b \in \real$. Then we can take the expected value of each side to get
\[P(|X-b| \geq a) \leq \frac{E((X-b)^2)}{a^2}\]
Taking $b = \mu$ and $a = k\sigma$ where $k > 0$, $\mu = E(X)$ and $\sigma^2 = \Var(X)$, we get Chebyshev's inequality.
\[P(|X - \mu| \geq k\sigma) \leq \frac{1}{k^2} \]
We can rewrite this as 
\[P(\mu - k\sigma < X < \mu + k\sigma) \geq 1 - \frac{1}{k^2} = \frac{k^2-1}{k^2}\]
This means that in the set $(\mu - k\sigma, \mu +k\sigma)$, there is at least $100\left(\frac{k^2-1}{k^2}\right)\%$ of the population.\\[2ex]
\textbf{Example.} Let $X$ be a random variable with
\[P(X = -1) = P(X = 1) = \frac{1}{8} \ \ P(X = 0) = \frac{6}{8}\]
We can see that $E(X) = 0$ and $\sigma^2 = \Var(X) = 1/4$. Let $k = 2$, then we can calculate
\[P(|X - \mu| \geq k\sigma) = P(|X| \geq 1) = \frac{1}{4}= \frac{1}{k^2}\]
This example shows Chebyshev's inequality cannot be improved since in this case it becomes an equality.
\begin{definition}
    An $n \times n$ matrix $U$ is non-negative definite if for any $n \times 1$ vector $c$, we have
    \[c'Uc \geq 0\]
    Furthermore, all of $U$ eigenvalues are non-negative.
\end{definition}
\noindent
\textbf{Example.}
Let 
\[U = \begin{bmatrix}
    1 & 0.5 \\
    0.5 & 1
\end{bmatrix}\]
Then 
\begin{align*}
    c'Uc &= \begin{bmatrix}
        c_1 & c_2
    \end{bmatrix}\begin{bmatrix}
        1 & 0.5 \\
        0.5 & 1
    \end{bmatrix}\begin{bmatrix}
        c_1 \\ c_2
    \end{bmatrix}\\
    &= \begin{bmatrix}
        c_1 + 0.5c_2 & 0.5c_1 + c_2
    \end{bmatrix}\begin{bmatrix}
        c_1 \\ c_2
    \end{bmatrix}\\
    &= c_1^2 + c_2^2 + c_1c_2\\
\end{align*}
Now it remains to prove that $c_1^2 + c_2^2 + c_1c_2 \geq 0$. This task is difficult so alternatively, it is much simpler to check that all eigenvalues are non-negative.
\[\det\begin{bmatrix}
    1 - \lambda & 0.5 \\
    0.5 & 1 - \lambda
\end{bmatrix}= (1-\lambda)^ - 0.25 = 0 \implies \lambda = 1 \pm \frac{1}{2}\]
So our eigenvalues are $\frac{1}{2}$, and $\frac{3}{2}$ which are both positive so $U$ is non-negative definite.\\[2ex]
\textbf{Note.} With the vector $p \times 1$ vector $X = [X_1, \ldots, X_p]$, we have 
\[XX' = \begin{bmatrix}
    X_1 & \cdots & X_p\\
\end{bmatrix}\begin{bmatrix}
    X_1 \\ \vdots \\ X_p
\end{bmatrix} = 
\begin{bmatrix}
    X_1^2 & X_1X_2 & \cdots & X_1X_p\\
    X_1X_2 & X_2^2 & \cdots & X_2X_p\\
    \vdots & \vdots & \ddots & \vdots\\
    X_1X_p & X_2X_p & \cdots & X_p^2
\end{bmatrix}\]
Then, 
\[E(XX') = \begin{bmatrix}
    E(X_1^2) & E(X_1X_2) & \cdots & E(X_1X_p)\\
    E(X_1X_2) & E(X_2^2) & \cdots & E(X_2X_p)\\
    \vdots & \vdots & \ddots & \vdots\\
    E(X_1X_p) & E(X_2X_p) & \cdots & E(X_p^2)
\end{bmatrix}\]
We can see that $XX'$ is non-negative definition since $c'XX'c = (X'c)'(X'c)$. Denote $(X'c)' = V$, then we have 
\[(X'c)'(X'c) = Y'Y = \begin{bmatrix}
    y_1 & \cdots & y_p
\end{bmatrix}\begin{bmatrix}
    y_1 \\ \vdots \\ y_p
\end{bmatrix} = y_1^2 + \cdots y_2^2\]
In conclusion, $XX'$ is non-negative definite and $E(XX')$ is non-negative definite since, for example, take $X = [X_1, X_2]^T$. Then 
\[XX' = \begin{bmatrix}
    X_1^2 & X_1X_2\\
    X_1X_2 & X_2^2
\end{bmatrix}\]
and $c'E(XX')c = E(c'XX'c) \geq 0$, so  
\[E(XX') = \begin{bmatrix}
    E(X_1^2) & E(X_1X_2)\\
    E(X_1X_2) & E(X_2^2)
\end{bmatrix}\]
is also non-negative definite.

\begin{theorem}[Cauchy-Shwarz Inequality]
    Let $X' = [X_1, \ldots, X_p]$ be a random vector. Define the matrix $U = E(XX')$. Notice that for any $p \times 1$ vector $c$ we have 
    \[c'Uc = E((c'X)^2) \geq 0\]
    Therefore the matrix $U$ is non-negative definite. Take $p = 2$ to see that 
    \[0 \leq \det(U) = E(X_1^2)E(X_2^2) - E((X_1X_2))^2\]
    Equality holds when $c'X = 0$, $c_1x_1 + c_2x_2 + \cdots + c_px_p = 0$. In other words, 
    \[E(XY)^2 \leq E(X^2)E(Y^2)\]
    and equality holds if there is a linear relationship between $X$ and $Y$.
    \[aX + bY = 0 \ \ a,b \in \real\]
\end{theorem}
\noindent
\textbf{Remark.} If we replace $X \rightarrow (X-\mu_X)$ and $Y \rightarrow (Y-\mu_Y)$, then we get
\[E[(X-\mu_X)(Y-\mu_Y)]^2 \leq E(X-\mu_X)E(Y-\mu_Y)\]
Notice that $E(X-\mu_X) = \Var(X)$ and $E[(X-\mu_X)(Y-\mu_Y)] = \Cov(X,Y)$, so can conclude 
\[\Cov(X,Y)^2 \leq \Var(X)\Var(Y)\]
Furthermore, 
\[\rho^2 = \frac{\Cov(X,Y)^2}{\Var(X)\Var(Y)} \leq 1\]
Where $\rho^2$ is the correlation between $X$ and $Y$. Therefore, when $X$ and $Y$ are linear and the Cauchy-Shwarz inequality becomes an equality, we have 
\[Y - \mu_Y = \beta(X - \mu_X)\]
for some constant $\beta$. Then, $\rho^2 = 1$.  
\section{Moment Generating Functions}
\begin{definition}
    Let $X$ be a random variable wtih cdf $F$, then the \emph{moment generating function} (m.g.f) of $X$ is defined as
    \[M(t) = E(e^{tX}) = \int_{-\infty}^\infty e^{tX}dF(X) < \infty\]
\end{definition}
\noindent
\textbf{Note.} Moment generating functions are unique, so with a (m.g.f) we can uniquely determine $F$ and vice-versa. 
\subsection{Properties of Moment Generating Functions}
\begin{enumerate}[label=(\roman*)]
    \item $M(0) = E(e^{0X}) = E(1) = 1$
    \item $\frac{d}{dt} M(t) = \int_{-\infty}^\infty xe^{tX}dF(X) = E(Xe^{tX}) \implies M'(0) = E(X)$
    \item $\frac{d^2}{dt^2} M(t) = \int_{-\infty}^\infty x^2e^{tX}dF(X) = E(X^2e^{tX}) \implies M''(0) = E(X^2)$. $E(X^2)$ is the 2nd moment of $X$.
    \item In general, 
    \[\frac{d^k}{dt^k} M(t) = \int_{-\infty}^\infty x^ke^{tx}dF(x)\]
    \[M^{(k)}(0) = E(X^k)\]
    \item We can use the Taylor series expansion of the moment generating function to get 
    \[M^{(k)}(t) = \sum_{k=0}^\infty \frac{M^{(k)}(0)}{k!}t^k = \sum_{k=0}^\infty \frac{E(X^k)}{k!}t^k\]
    Then if we multiply by $k!$ to get 
    \[E(X^k) = (\text{Multiple of $t^k$}) \times k! = \frac{M^{(k)}(0)}{k!} \cdot k! = M^{(k)}(0)\]
\end{enumerate}
\noindent
\textbf{Example.} Recall the gamma distribution 
\[f(x) = \frac{\beta^\alpha}{\Gamma(\alpha)}x^{\alpha-1}e^{-\beta x}I(x > 0)\]
With 
\begin{align*}
    \Gamma(\alpha) &= \int_0^\infty e^{-x}x^{\alpha-1}dx\\ 
    &= \left[x^{\alpha -1}e^{-x}\right]_0^{\infty} + \int_0^\infty e^{-x}(\alpha - 1)x^{\alpha -2}dx\\ 
    &= (\alpha - 1)\int_0^\infty e^{-x}(\alpha - 1)x^{\alpha -2}dx\\
    &= (\alpha - 1)\Gamma(\alpha - 1)
\end{align*}
Then, we have 
\[\Gamma(1) = \int_0^\infty e^{-x}dx = 1\]
\[\Gamma(2) = \int_0^\infty xe^{-x}dx = \left[-xe^{-x}\right]_0^\infty + \int_0^\infty e^{-x}dx = 1\]
So, 
\[\Gamma(3) = 2\Gamma(2) = 2, \ \Gamma(4) = 3 \Gamma(3) = 6 = 3!, \ \Gamma(5) = 4!, \ \ldots\]
Therefore,
\[\alpha \in \nat \implies \Gamma(\alpha) = (\alpha-1)!\]
If we set $x = \beta u$, we get 
\[\Gamma(\alpha) = \int_0^\infty e^{-\beta u}\beta^{\alpha -1}u^{\alpha-1}\beta du = \beta^\alpha \int_0^\infty e^{-\beta u}u^{\alpha -1}du\]
This gives us the equation for the Laplace transform for $u^{\alpha -1}$. 
\[\frac{\Gamma(\alpha)}{\beta^\alpha} = \int_0^\infty e^{\beta u}u^{\alpha - 1}du = \mathcal{L}(u^{\alpha -1})\]
This can be useful for calculating integrals, such as 
\[\int_0^\infty e^{-3x}x^5dx = \frac{5!}{3^6}\]
If we multiply the Laplace transform by $\frac{\beta^\alpha}{\Gamma(\alpha)}$, 
\[\frac{\Gamma(\alpha)}{\beta^\alpha} \cdot \frac{\beta^\alpha}{\Gamma(\alpha)} = \int_0^\infty \frac{\beta^{\alpha}}{\Gamma(\alpha)}e^{-\beta u}u^{\alpha -1}du = 1 \]
Since this function is positive and integrates to 1, it is a p.d.f. 
\[f(u) = \frac{\beta^{\alpha}}{\Gamma(\alpha)}e^{-\beta u}u^{\alpha -1}\]
Thus, this is the Gamma distribution. If we take $\alpha = r/2$, $\beta = 1/2$. We get a special case of the Gamma distribution 
\[f(u) = \frac{1}{\Gamma\left(\frac{r}{2}\right)2^{r/2}}e^{-u/2}u^{\frac{r}{2}-1}\]
This is the chi-squared distribution with $r$ degrees of freedom. Now we can find the moment generating function for the Gamma distribution 
\begin{align*}
    M(t) = E(e^{tx}) &= \int_0^\infty \frac{\beta^\alpha}{\Gamma(\alpha)}e^{-\beta x}x^{\alpha -1}e^{tx}dx\\
    &= \frac{\beta^\alpha}{\Gamma(\alpha)}\int_0^\infty e^{-(\beta-t) x}x^{\alpha -1}e^{tx}dx\\
    &= \frac{\beta^\alpha}{\Gamma(\alpha)}\cdot \frac{\Gamma(t)}{(\beta-t)^\alpha}\\
    &= \beta^\alpha(\beta-t)^{-\alpha} = \left(1 - \frac{t}{\beta}\right)^{-\alpha} 
\end{align*}
We can also calculate the expected value and variance
\[M'(t) = \frac{\alpha}{\beta}\left(1 - \frac{t}{\beta}\right)^{-\alpha - 1} \implies M'(0) = \frac{\alpha}{\beta} = 1\]
\[M''(t) = \frac{\alpha}{\beta} \cdot -\frac{1}{\beta}(-\alpha -1)\left(1 - \frac{t}{\beta}\right)^{-\alpha -2} \implies M''(0) = \frac{\alpha(\alpha+1)}{\beta^2}\]
\[\Var(X) = E(X^2)-E(X)^2 = \frac{\alpha(\alpha+1)}{\beta^2} - \frac{\alpha^2}{\beta^2} = \frac{\alpha}{\beta^2}\]
Similarly, for the chi-squared distribution, we have $\alpha = r/2$ and $\beta = 1/2$,
\[M(t) = (1-2t)^{-\frac{r}{2}}\]
\[E(X\sim \chi^2(r)) = \frac{\alpha}{\beta}=\frac{r/2}{1/2} = r\]
\[\Var(X \sim \chi^2(r)) = \frac{\alpha}{\beta^2} = \frac{r/2}{1/4} = 2r \]
\subsection{Applications of Moment Generating Functions}
\begin{enumerate}
    \item Suppose we have $k$ random variables $X_1 \sim \chi^2(r_1), \ldots, X_k \sim \chi^2(r_k)$.  If they're independent, then we can define $Y = \sum\limits_{i=1}^k X_i$, and calculate the moment generating function 
    \begin{align*}
        E(e^{tY}) &= E\left(\exp\left(t\sum_{i=1}^k X_i\right)\right)\\
        &= E(e^{tX_1} \cdot e^{tX_2} \cdots e^{tX_k})\\
        &= E(e^{tX_1}) \cdot E(e^{tX_2}) \cdots E(e^{tX_k})\\
        &= (1-2t)^{-r_1/2} \cdot (1-2t)^{-r_2/2} \cdots (1-2t)^{-r_k/2}\\
        &= (1-2t)^{-(r_1 + r_2 + \cdots + r_k)/2}
    \end{align*}
    Therefore, $Y \sim \chi^2(r_1 + r_2 + \cdots + r_k)$
    \item Let $Z \sim N(0,1)$ with pdf 
    \[f(z) = \frac{1}{2\sqrt{\pi}}e^{-z^2/2}\]
    We know that since $f(z)$ is a p.d.f,
    \[\int_{-\infty}^\infty \frac{1}{2\sqrt{\pi}}e^{-z^2/2}dz = 1\]
    We can calculate the m.g.f 
    \begin{align*}
        M(t) = E(e^{tZ}) &= \int_{-\infty}^\infty \frac{1}{\sqrt{2\pi}}e^{-z^2/2}e^{tz}dz\\
        &= \int_{-\infty}^\infty \frac{1}{\sqrt{2\pi}}e^{\frac{-(z-t)^2}{2}}e^{t^2/2}dz\\
        &= e^{t^2/2}\int_{-\infty}^\infty \frac{1}{\sqrt{2\pi}}e^{\frac{-(z-t)^2}{2}}dz\\
        &= e^{t^2/2} \cdot 1
    \end{align*} 
    The integral is 1 since it is the integral of the p.d.f of the normal distrubtion with $N(t, 1)$. Recall that the p.d.f for a general normal distribution $N(\mu, \sigma^2)$ is 
    \[f(x) = \frac{1}{\sigma\sqrt{2\pi}}e^{-(x-\mu)^2/2\sigma^2}\]
    Using the Taylor expansion for $e^x$ we can rewrite the m.g.f as 
    \[E(e^{tZ}) = e^{t^2/2} = \sum_{k=0}^\infty \frac{(t^2/2)^k}{k!} = \sum_{k=0}^\infty \frac{t^{2k}}{k!2^k}\]
    We say earlier that $E(X^k) = \text{Some multiple of $t^k$} \cdot k!$. So this means that $E(X^{2k+1}) = 0$ since $2k+1$ is odd but we have only even multiples of $t^k$. Therefore, all odd moments have 0 expecation. All even moments are 
    \[E(X^{2k}) = \frac{(2k)!}{k!2^k}\]
\end{enumerate}
\noindent
\textbf{Example.} Let $Z \sim N(0,1)$, find the m.g.f for $Z^2$. \\[2ex]
\textbf{Solution.}
\begin{align*}
    M(t) &= E(e^{tZ^2})\\
    &= \int_{-\infty}^\infty \frac{1}{\sqrt{2\pi}}e^{-z^2/2}e^{tz^2}dz\\
    &= \int_{-\infty}^\infty \frac{1}{\sqrt{2\pi}}e^{-z^2(-t + 1/2)}dz\\
    &= (1-2t)^{-1/2} = \frac{1}{\sqrt{1-2t}}
\end{align*}
Note that in general if $Z \sim N(0, 2\alpha)$, we have 
\[\int_{-\infty}^\infty \frac{1}{\sqrt{2\pi}}e^{-\alpha z^2}dz = \frac{1}{\sqrt{2\alpha}}\]
To summarize,
\[Z \sim N(0,1) \implies M(t) = e^{t^2/2}\]
\[Y = Z^2 \implies M_Y(t) = (1-2t)^{-1/2} \rightarrow \chi^2(1)\]
\[M_{\chi^2}(t) = (1-2t)^{-r/2}\]
\[Z \sim N(0,1) \implies Z^2 \sim \chi^2(1)\]
\noindent
\textbf{Example.}
Let $X_1, \ldots, X_n$ be independent random variables with $X_i \sim N(\mu_i, \sigma_i^2)$. What is the distribution for $Y = \sum\limits_{i=1}^n X_i$?\\[2ex]
\textbf{Solution.} 
We can calculate the m.g.f for $X_i$ 
\[M_{X_i}(t) = E(e^{t(\sigma_i z + \mu_i)}) = e^{t\mu_i} E(e^{(t\sigma_i)z}) = e^{t\mu_i + t^2\sigma_i^2/2} \]
Then, the m.g.f for $Y$ is
\begin{align*}
    M_Y(t) &= E(e^{tY})\\
    &= E \left(\exp\left(t\sum a_iX_i\right)\right)\\
    &= E(e^{ta_iX_1} \cdot e^{ta_iX_2} \cdots e^{ta_iX_n})\tag{Independence}\\
    &= \exp\left(ta_1\mu_1 + \frac{t^2a_1^2\sigma_n^2}{2}\right) \cdots \exp\left(ta_n\mu_n + \frac{t^2a_n^2\sigma_n^2}{2}\right)\\
    &= \exp\left(t(a_1\mu_1 + \cdots + a_n\mu_n) + \frac{t^2}{2}(a_1^2\sigma_1^2 + \cdots + a_n^2\sigma_n^2)\right)
\end{align*}
Notice that the m.g.f for the general normal distribution is $M(t) = \exp(t\mu + t^2\sigma^2/2)$. So in this case we have $\mu = a_1\mu_1 + \cdots + a_n\mu_n$, and $\sigma^2 = a_1^2\sigma_1^2 + \cdots + a_n^2\sigma_n^2$. Therefore, 
\[Y = \sum_{i=1}^n a_iX_i \sim N\left(\sum_{i=1}^n a_i\mu_i, \sum_{i=1}^n a_i^2\sigma_i^2\right)\]
\subsection*{Binomial Distribution}
Recall a Bernoulli random variable $X$ is a random variable with $P(X = 1) = p$, and $P(X = 0) = 1 - p = q$, then $X \sim \Bern(p)$, $E(X) = 0q + 1p = p$, $E(X^2) = 0^2q + 1^2p = p$, $\Var(X) = p - p^2 = p(1 - p) = pq$. Let $X_1, \ldots X_n \iid \Bern(p)$, then $Y = \sum\limits_{i=1}^n X_i$ is the number of 1's observed, or in other words the number of successes. Then, we have 
\[P(Y = k) = {n \choose k}p^k (1-p)^{n-k}\]
We can prove this by finding the m.g.f for $Y$, 
\begin{align*}
    E(e^{tY}) &= E(e^{tX_1})\cdots E(e^{tX_n})\\
    &= (pe^t + q)^n
\end{align*}
\begin{align*}
    E(e^{tY}) &= \sum_{k=0}^n {n \choose k} p^k(1-p)^{n-k}e^{tk}\\
    &= \sum_{k=0}^n {n \choose k} (pe^t)^k(1-p)^{n-k} 
    &= (pe^t + q)^n
\end{align*}
Notice that from the binomial expansion we have 
\[(a+b)^n = \sum_{k=0}^n {n \choose k}a^kb^{n-k}\]
and replace $a$ with $pe^t$ and $b$ with $q = 1-p$. In conclusion, the sum of Bernoulli random variables $Y = \sum\limits_{i=1}^n X_i$ is a binomial random variable $Y \sim \Bin(n,p)$.\\[2ex]
\textbf{Example.} Let $X_1 \sim \Bin(m,p)$, $X_2 \sim \Bin(n,p)$ be independent random variables. Then $X_1 + X_2 \sim \Bin(m+n, p)$. We can show this using the moment generating function.
\[M_{X_1 + X_2}(t) = E(e^{t(X_1 + X_2)}) = E(e^{tX_1})E(e^{tX_2}) = (pe^t+q)^{m+n}\]
\section{Holder, Lyapunov and Minkowski Inequalities}
\begin{corollary}
    If $P(X \geq 0) = 1$ and $E(X) = \mu$, then 
    \[P(X \geq 2\mu) \leq 0.5\]
\end{corollary}
\begin{proof}
    Using Markov's Inequality, 
    \[P(X \geq 2\mu) \leq \frac{E(X)}{2\mu} = \frac{1}{2}\]
\end{proof}
\begin{lemma}
    If $\alpha \geq 0$, $\beta \geq 0$, and
    \[\frac{1}{p} + \frac{1}{p} = 1, \ p>1, \ q > 1\]
    then 
    \[0 \leq \alpha \beta \leq \frac{\alpha^p}{p} + \frac{\beta^p}{q}\]
\end{lemma}
\begin{proof}
    If $\alpha\beta=0$, then the inequality holds trivially. Therefore, let $\alpha > 0$, $\beta > 0$. Then, define for $t > 0$
    \[\phi(t) = \frac{t^p}{p} + \frac{t^{-q}}{q}\]
    Differenting this function we get 
    \[\phi'(t) = t^{p-1} - t^{-q-1}\]
    We can see that $\phi'(1) = 0$, $\psi'(t) < 0$ when $t \in (0,1)$, and $\psi'(t) > 0$ when $t > 1$. Thus, $t$ minimizes $\phi$ on $(0,\infty)$. Set $t = \frac{a^{1/q}}{\beta^{1/p}}$ to get 
    \[\frac{\alpha^{p/q}}{p\beta} + \frac{\alpha^{-1}}{q\beta^{-q/p}} \geq 1\]
    Multyipling both sides by $\alpha\beta$ and using 
    \[p/q + 1 = p \ \text{ and } \ q/p + 1 = q\]
    we get
    \[\alpha\beta \leq \frac{\alpha^{p/q+1}}{p} + \frac{\beta}{q\beta^{-q/p}} = \frac{\alpha^p}{p} + \frac{\beta^q}{q}\]
\end{proof}
\begin{theorem}[Holder's Inequality]
    Let $X$ and $Y$ be two random variables and 
    \[\frac{1}{p} + \frac{1}{q} = 1, \ p > 1, \ q > 1\]
    We have 
    \[E(|XY|) \leq (E(|X|^p))^{1/p}(E(|Y|^q))^{1/q}\]
\end{theorem}
\begin{proof}
    In the case that $E(|X|^p)E(|Y|^q) = 0$, the result follows. Otherwise, from Lemma 2.3.1, take 
    \[\alpha = \frac{|X|}{(E(|X|^p))^{1/p}}, \ \beta = \frac{|Y|}{(E(|Y|^q))^{1/q}}\]
    Then using 
    \[\alpha\beta \leq \frac{\alpha^p}{p} + \frac{\beta^q}{q}\]
    we get
    \[\frac{|XY|}{(E(|X|^p))^{1/p}(E(|Y|^q))^{1/q}} \leq \frac{|X|^p}{pE(|X|^p)} + \frac{|Y|^q}{qE(|Y|^q)}\]
    Now taking the expected value, we have
    \[\frac{E(|X|^p)}{pE(|X|^p)} + \frac{E(|Y|^q)}{qE(|Y|^q)} = \frac{1}{p} + \frac{1}{q} = 1\]
    Therefore,
    \[\frac{E(|XY|)}{(E(|X|^p))^{1/p}(E(|Y|^q))^{1/q}} \leq 1 \implies E(|XY|) \leq (E(|X|^p))^{1/p}(E(|Y|^q))^{1/q}\]
\end{proof}
\begin{theorem}[Minkowski's Inequality]
    For $p\geq 1$, we have 
    \[E(|XY|) \leq (E(|X|^p))^{1/p}(E(|Y|^q))^{1/q}\]
\end{theorem}
\begin{proof}
    Since $|X + Y| \leq |X| + |Y|$, the case that $p=1$ is obvious. Let $p > 1$, choose $q$ such that 
    \[\frac{1}{p} + \frac{1}{q} = 1\]
    Then, use Holder's inequality to write
    \begin{align*}
        E(|X + Y|^p) &= E(|X+Y||X+Y|^{p-1})\\
        &\leq E(|X||X+Y|^{p-1}) + E(|Y||X+Y|^{p-1})\\
        &\leq (E(|X|^p))^{1/p}(E(|X+Y|^{(p-1)q}))^{1/q} + (E(|Y|^p))^{1/p}(E(|X+Y|^{(p-1)q}))^{1/q}\\
        &= (E(|X|^p))^{1/p}(E(|X+Y|^{p}))^{1/q} + (E(|Y|^p))^{1/p}(E(|X+Y|^{p}))^{1/q}\\
        &= (E|X+Y|^p)^{1/q}((E(|X|^p))^{1/p} + (E(|Y|^p))^{1/p})
    \end{align*}
    Now we can divide both sides by $(E(|X+Y|^p))^{1/q}$ to get
    \[\frac{E(|X+Y|^p)}{(E|X+Y|^p)^{1/q}} = (E|X+Y|^p)^{1 - 1/q} = (E|X+Y|^p)^{p}\]
    Thus, 
    \[ (E|X+Y|^p)^{p} \leq (E(|X|^p))^{1/p} + (E(|Y|^p))^{1/p}\]
\end{proof}
We can define 
\[||X||_p - E(|X|^p)^{1/p}\]
And we can define a space of random variables whose $p$th moment exists as 
\[\mathcal{X} = \{X: E|X|^p < \infty\}\]
Then we can calculate a metric on $\mathcal{X}$ as 
\[d(X,Y) = (E|X-Y|^p)^{1/p}\]
We can confirm that $d(X,Y)$ is a metric by confirming the axioms 
\begin{enumerate}[label=(\roman*)]
    \item $d(X,X) = 0$
    \item $d(X,Y) = d(Y,X)$
    \item $d(X,Z) \leq d(X,Y) + d(Y,Z)$
\end{enumerate}
The triangular inequality is Minkowski's inequality which we proved previously. Thus we have the Hilbert Space $(\mathcal{X}, d)$. 
\[\langle X,Y\rangle = \int X(\omega)Y(\omega)d\omega\]
this is an inner product space. A special case of Mikowski's inequality is when $p=1$ and $p=2$, 
\[(E|X+Y|^2)^{1/2} \leq (E(X^2))^{1/2} + (E(Y^2))^{1/2}\]
\[E|X+Y| \leq E(X) + E(Y)\]
\begin{theorem}[Jensen's Inequality]
    Let $\phi(x)$ be a convex function, then 
    \[\phi(E(X)) \leq E(\phi(X))\]
\end{theorem}
The proof for this is simple since if the function is convex, its derivatives are increasing, thus the secant lines from any points $x_1,x_2$ on the function are above the curve, therefore the average of the function and the average of the secant lines is 
\[\phi\left(\frac{x_1 + x_2}{2}\right) \leq \frac{f(x_1)+f(x_2)}{2}\] 
\begin{theorem}[Lyapunov's Inequality]
    If $r > s > 0$, then 
    \[||X|_r = (E|X|^r)^{1/r} \geq (E|X|^s)^{1/s} = ||X||_s\]
\end{theorem}
\begin{proof}
    Define $g(x) = |x|^u$ with $u > 1$ so that $g$ is convex. Then, we know that $r/s > 1$ since $r > s > 0$. From Jensen's inequality, we can take $u = r/s > 1$ to get
    \[E(g(x)) \geq g(E(X)) \iff E(|X|^{r/s}) \geq (E|X|)^{r/s}\]
    Then we can rewrite this as
    \[E(|X|^r) = E\left[(|X|^s)^{r/s}\right] \geq (E|X|^s)^{r/s}\]
    Replacing $|X|$ with $|X|^s$, 
    \[E(|X|^r) \geq (E|X|^s)^{r/s}\]
    We can take the $r$th root of each side and this inequality will still hold, 
    \[E(|X|^r)^{1/r} \geq (E|X|^s)^{1/s}\]
    as required.
\end{proof}
\chapter{Induced Probability Measures}
Recall that a random variable $X$ is a map from the sample space $\Omega$ to the real number line, so $X: \Omega \rightarrow \real$.\\[2ex]
\textbf{Example.} Consider the flip of a coin with $\Omega = \{H,T\}$. Let $X$ be the number of heads observed, then $X(H) = 1$, and $X(T) = 0$.\\[2ex]
\textbf{Another Example.} Consider the result of rolling 2 die, so 
\[\Omega = \{(1,1), (1,2), (1,3), \ldots, (6,5), (6,6)\}\] 
So $|\Omega| = 36$. Then define an $\omega = (x,y) \in \Omega$ as the result of rolling the first die $x$ and the second die $y$, and the random variable $X$ to be $X(\omega) = x + y$. So $X(2,3) = 5$, $X(6,6) = 12$ etc. We can define a different random variable $Y(x,y) = |x-y|$. We can ask questions about probabilities on $Y$, such as what is the probability $Y = 1$? 
\[P(Y = 1) = P(\{\omega: Y(\omega) = 1\}) = P(\{(1,2),(2,1),(2,3),(3,2), \ldots\}) = \frac{10}{36}\]
So $\{\omega: Y(\omega) = 1\} = Y^{-1}(\{1\})$. In otherwords the probability of an event $E$ occuring can be written as 
\[P_Y(E) = P(Y^{-1}(E))\]  
Where $P_Y$ is called the \emph{induced probability measure} by $Y$. We can check that $P_Y$ satisfies all conditions of a probability measure.
\[P_Y(\real) = 1\]
If $E_i$'s are disjoint, then
\[P_Y\left(\bigcup_{i=1}^\infty E_i\right) = \sum_{i=1}^\infty P_Y(E_i)\]
since 
\[Y^{-1}\left(\bigcup_{i=1}^\infty E_i\right) = \bigcup_{i=1}^\infty Y^{-1}(E_i)\]
\noindent
\textbf{Example.} Flip 2 coins and let $X$ be the number of heads observed. 
\[\Omega = \{HH, HT, TH, TT\}\]
and 
\[X(HH) = 2, X(HT) = X(TH) = 1, X(TT) = 0\]
So $X: \Omega \rightarrow \{0,1,2\} \subset \real$. We can define the induced probability measure $P_X$ by
\[P_X(\{0\}) = P(X=0) = P(\{TT\}) = \frac{1}{4}\]
\[P_X(\{1\}) = P(X=1) = P(\{TH, HT\}) = \frac{1}{2}\]
\[P_X(\{1,2\}) = P(X=0 \text{ or } X = 1) = P(\{TH, HT, HH\}) = \frac{3}{4}\]
\[P_X(\{0,2\}) = P(X=0) + P(X = 2) = P(\{TT, HH\}) = \frac{1}{2}\]
\[P_X(\{0,1,2\}) = 1\]
\section{Cumulative Distribution Functions}
We can know look at a more exact definition of c.d.f's. 
\begin{definition}[Cumulative Distribution Functions]
    If $X$ is a random variable $X: \Omega: \real$, then the c.d.f of $X$ is
    \[F_X(x) = P(X \leq x) = P(\{\omega: X(\omega) \leq x\}) = P_X((-\infty, x])\] 
\end{definition}
\subsection{Properties of Cumulative Distribution Functions}
\begin{theorem}
    
    Let $F$ be a c.d.f, then
    \begin{enumerate}[label=(\roman*)]
        \item $F(x) \geq 0$ and $F(x)$ is non-decreasing.
        \item $\lim\limits_{x\rightarrow t^+}F(x) = F(t)$
        \item $F(-\infty) = 0$ and $F(-\infty)= 1$.
    \end{enumerate}
\end{theorem}
\begin{proof}
    \begin{enumerate}[label=(\roman*)]
        \item  To prove $F$ is non-decreasing, we need to show 
        \[F(x + h) - F(x) \geq 0, \ \forall h \geq 0\]
        \[F(x + h) - F(x) = P(x < X \leq x+h) \geq 0\]
        Since probability measures are positive, therefore $F$ is non-decreasing.
        \item Consider 
        \[E_n = (-\infty, x + h_n)\]
        With $h_n \downarrow 0$, for example $h_n = \frac{1}{n} \downarrow 0$.  Then, $E_n$ is decreasing so the limit is 
        \[\lim_{n\rightarrow \infty} E_n = \bigcap_{n=1}^\infty E_n = (-\infty, x]\]
        Thus, from \hyperref[lem:1.5.2]{Lemma 1.5.2}, we have
        \[\lim_{n\rightarrow \infty} P(E_n) = P\left(\lim_{n\rightarrow\infty} E_n\right) = P\left((-\infty, x]\right) = F(x)\]
        Therefore we showed that 
        \[\lim_{n\rightarrow\infty} F(x + h_n) = F(x)\]
        So $F$ is right continuous.
        \item We know $F(N) - F(-N) = P(-N < X \leq N) = P((-N,N])$ for any, $N$, so 
        \[\lim_{N \rightarrow \infty} = (-\infty,\infty) \implies P((-N,N])\uparrow 1\]
        Therefore,
        \[\lim_{N\rightarrow \infty} [F(N) - F(-N)] = 1 \]
        Thus $F(\infty) = 1$ and $F(-\infty) = 0$.
    \end{enumerate}
\end{proof}
\noindent
\textbf{Remark.} A distrbiution function $F$ is continuous at $x \in \real$ if and only if $P(X = x) = 0$, since 
\[P(X=x) = F(x) - \lim_{x \rightarrow x^-}F(x) = \text{The size of the jump at $x$}\]
\begin{definition}
    A random variable $X$ is of continuous type if $F(x) = P(X \leq x)$ is a continuous function. 
\end{definition}
\noindent
\textbf{Example.} Let 
\[F(x) = \frac{1}{2}I(x \in [0,1)) + \frac{2x}{3}I\left(x \in [1, 3/2)\right) + I\left(x \in [3/2, \infty)\right)\]
This c.d.f is not continuous since there is a jump at $x = 0$ and $x = 1$, so we have 
\[P(X = 0) = F(0) - \lim_{x\rightarrow 0^-} F(x) = \frac{1}{2} - 0\]
\[P(X = 1) = \frac{3}{2} - \frac{1}{2} = \frac{1}{6}\]
Recall that a set is called \emph{countable} if we can define a bijection from the natural numbers to the set. For example 
\[f:\nat \mapsto \{2,4,6,8,\ldots\}  \implies f(n) \coloneqq 2n\]
We say a set is at most countable if it is finite or countable. 
\begin{lemma}
    If $F$ is a c.d.f, then the number of discontinuities of $F$ is at most countable.
\end{lemma}
\begin{proof}
    Define 
    \[p(x) \coloneqq F(x) - F(x^-) = P(X = x)\]
    Let $D$ be the set of discontinuities of $F$, so 
    \[D = \{x: p(x) > 0\}\]
    We need to show that $D$ is at most countable. Let 
    \[D_n \coloneqq \left\{x: \frac{1}{n+1} < p(x) \leq \frac{1}{n}\right\}\]
    Then, 
    \[\bigcup_{n=1}^\infty D_n = D\]
    We must prove that $D_n$ is finite. $D_n$ is bounded below by $\frac{1}{n+1}$, so $|D_n|$ is at most $n$, since if we have $n+1$ points, 
    \[P(D_n) > \frac{n+1}{n+1} = 1\]
    which is a contradiction. Therefore, $D$ is at most countable.
\end{proof}
\begin{lemma}
    Let $X$ be a random variable with c.d.f $F$ and $p(x) = F(x) - F(x^-) = P(X=x)$. Let $D = \{x_1,x_2, \ldots\} $ be the set of discontinuities of $F$. Define the step function 
    \[G(x) = \sum_{n=1}^\infty p(x_i)I(x \geq x_i)\]
    Then $H(x) = F(x) - G(x)$ is non-decreasing and continuous on $\real$.
\end{lemma}
\textbf{Note.} We call $G$ a step function since it increases in discrete intervals and is constant in between. \\[2ex]
\begin{proof}
    Obviously $H$ is right continuous since $F(x)$ and $G(x)$ are right continuous. We want to show that $H$ is also left continuous. Not that if $x' < x$, then 
    \[H(x) - H(x') = F(x) - F(x') - G(x) + G(x')\]
    As $x' \uparrow x$, then $F(x) - F(x')$ converges to the size of jummp of $F$ at $x$, and $G(x) - G(x')$ converges to the size of the jump of $G$ at $x$. The size of jump in both cases is $p(x)$ which shows that $H(x) - H(x') \rightarrow 0$ as $x'\uparrow x$. Therefore $H$ is continuous. Now we want to show that $H$ is non-decreasing. Note that
    \[\sum_{x' < x_n \leq x} p(x_n)\leq P(x' < X \leq x) = F(x) - F(x')\]
    We want to show that $H(x) \geq H(x')$.
    \begin{align*}
        H(x') &= F(x') - G(x')\\
        \implies H(x) - H(x') &= F(x) - F(x') - (G(x) - G(x'))\\
        &= F(x) - F(x') - \sum_{x' < x_n \leq x} p(x_n)\\ 
    \end{align*}    
    We know that $\sum\limits_{x' < x_n \leq x} p(x_n) \leq F(x) - F(x')$, therefore $H(x) - H(x') \geq 0$. Thus $H$ is non-decreasing. 
\end{proof}
\begin{theorem}
    Let $F$ be a c.d.f, then there exists two c.d.f's $F_d$, $F_c$ where $F_d$ is a discrete step function and $F_c$ is a continuous function such that
    \[F = \alpha F_d + (1- \alpha)F_c\]
    for some $\alpha \in [0,1]$, and 
    \[F_d = \frac{G(x)}{\alpha}\]
    \[F_c = \frac{H(x)}{1 - \alpha}\]
\end{theorem}
\noindent
\textbf{Example.} Find $F_c$, $F_d$, and $\alpha$ for the following c.d.f
\[F(x) = \frac{1}{2}I(0 \leq x < 1) + \frac{2x}{3}I(1 \leq x < 3/2) + I(x \geq 3/2)\]
\textbf{Solution.} Consider the set of discontinuities of $F$, we have $\{x_1 = 0, x_2 = 1\}$. Then,
\begin{align*}
    p(x_1) &= p(0) = \frac{1}{2}\\
    p(x_2) &= p(1) = \frac{2}{3} - \frac{1}{2} = \frac{1}{6}
\end{align*}
Then we can define $G(x)$, 
\[G(x) = \sum_{x_i \leq x} G(x)\]
If $x < 0$, then $G(x) = 0$, similarly 
\[0 \leq x < 1 \implies G(x) = \frac{1}{2}\]
\[1 \leq x \implies G(x) = \frac{1}{2} + \frac{1}{6} = \frac{2}{3}\]
So we can define $G(x)$ with indicator functions 
\[G(x) = \frac{1}{2}I(0 \leq x < 1) + \frac{2}{3}I(1 \leq x)\]
We can see that $G(x)$ is not a c.d.f since $G(\infty) \neq 1$, then our $\alpha$ is 
\[\alpha = G(\infty) = \frac{2}{3} \implies 1 - \alpha = \frac{1}{3}\]
To find $H(x)$, we can write $G(x)$ and $F(x)$ as 
\[F(x) = \begin{cases}
    0 & x < 0\\
    \frac{1}{2} & 0 \leq x < 1\\
    \frac{2x}{3} & 1 \leq x < \frac{3}{2}\\
    1 & x \geq \frac{3}{2}
\end{cases}\]
\[G(x) = \begin{cases}
    0 & x < 0\\
    \frac{1}{2} & 0 \leq x < 1\\
    \frac{2}{3} & 1 \leq x < \frac{3}{2}\\
    \frac{2}{3} & x \geq \frac{3}{2}
\end{cases}\]
Then, 
\[H(x) = F(x) - G(x) = \begin{cases}
    0 & x < 0\\
    0 & 0 \leq x < 1\\
    \frac{2x}{3} - \frac{2}{3} = \frac{2}{3}(x-1) & 1 \leq x < \frac{3}{2}\\
    \frac{1}{3} & x \geq \frac{3}{2}
\end{cases}\]
Now we can find $F_d$ and $F_c$, 
\[F_d(x) = \frac{G(x)}{\alpha} = \frac{3G(x)}{2} = \begin{cases}
    0 & x < 0\\
    \frac{3}{4} & 0 \leq x < 1\\
    1 & x \geq 1
\end{cases}\]
\[F_c(x) = \frac{H(x)}{1 - \alpha} =  3H(x) = \begin{cases}
    0 & x <1\\
    2(x-1) & 1 \leq x < \frac{3}{2}\\
    1 & x \geq \frac{3}{2}
\end{cases}\]
Notice that the p.d.f
\[\frac{dF_d(x)}{dx} = \begin{cases}
    \frac{3}{4} & x = 0\\
    \frac{1}{4} & x = 1
\end{cases}
\]
This is the p.d.f of a Bernoulli random variable with $p = 1/4$, and 
\[\frac{dF_c(x)}{dx} = \begin{cases}
    2 & 1 \leq x < \frac{3}{2}\\
    0 & \text{otherwise}
\end{cases}\]
This is the p.d.f of a uniform random variable on $[1,3/2]$. Therefore, $F$ is a linear combination of a Bernoulli random variable and a uniform random variable. We can calculate the expected value 
\[E(X) = \int xdF(x) = \alpha \int x dF_d(x) + (1-\alpha)\int x dF_c(x)\]
\textbf{Example.} Let $X$ be a random variable with continuous c.d.f $F$. Find the distrbiution for 
\begin{enumerate}[label=(\roman*)]
    \item $U = F(X)$
    \item If $U \sim [0,1]$, then $X \eqd F^{-1}(U)$
    \item Use (i) and (ii) to show that $X = -\ln U$ has an exponentional distrbution. 
\end{enumerate}\
\textbf{Note.} If $U \sim \Unif(0,1)$, then 
\[U \eqd 1 - U\]
since the c.d.f for $1 - U$ is 
\[G(t) = P(1 - U \leq t) = P(U \geq 1 - t) = \int_{1-t}^1 du = t\]
\textbf{Solution.} 
\begin{enumerate}[label=(\roman*)]
    \item We have
    \[G(u) = P(F(X) \leq u) = P(F^{-1}(F(X)) \leq F^{-1}(u)) = F(F^{-1}(u)) = u\]
    Therefore, $g(u) = G'(u) = 1$ for $u \in [0,1]$.
    \item Notice that 
    \begin{align*}
        P(F^{-1}(U) \leq x) &= P(F(F^{-1}(U)) \leq F(s))\\
        &= P(U \leq F(x))\\
        &= \int_0^{F(x)} du = F(x)
    \end{align*}
    \item Now using (i), (ii)
    \begin{align*}
        F(x) &= P(-\ln U \leq x)\\
        &= P(U \geq e^{-x})\\
        &= 1 - P(U \leq e^{-x})\\
        &= 1 - e^{-x}\\
        \implies f(x) &= F'(x) = e^{-x}I(x \geq 0)
    \end{align*}    
\end{enumerate}



\section{Conditional Probability Measure}
\begin{definition}
    Let $P(X \leq x) = F(x)$ be continuous. If there exists a non-negative function $f$ such that 
    \[P(X \leq x) = \int_{-\infty}^x f(t)dt\]
    Then $f$ is called the probability density function (p.d.f). 
\end{definition}

\begin{definition}
    Let $(\Omega, \mathcal{F}, P)$ be a probability space and $B \in \mathcal{F}$ with $P(B) > 0$. The conditional probability measure given $B$ as 
    \[Q_B(A) = P(A|B) = \frac{P(A \cap B)}{P(B)}\]
\end{definition}
\begin{theorem}
    $Q_B$ is a probability measure on $(\Omega, \mathcal{F})$.
\end{theorem}
\begin{proof}
    We can check each property of probability measures
    \begin{enumerate}[label=(\roman*)]
        \item $Q_B(\emptyset) = 0$ since 
        \[Q_B(\emptyset) = \frac{P(\emptyset \cap B)}{P(B)} = \frac{P(\emptyset)}{P(B)}= 0\]
        \item The sum of disjoint sets is
        \begin{align*}
            Q_B\left(\bigcup_{i=1}^\infty A_i\right) &= \frac{P(B \cap (A_1 \cup A_2 \cup \cdots))}{P(B)}\\
            &= \frac{P((B \cap A_1) \cup (B \cap A_2) \cup \cdots)}{P(B)}\\
            &= \sum_{i=1}^\infty \frac{P(B \cap A_i)}{P(B)}\\
            &= \sum_{i=1}^\infty Q_B(A_i)
        \end{align*}
        \item $Q_B(\Omega) = 1$ since
        \[Q_B(\Omega) = \frac{P(\Omega\cap B)}{P(B)} = \frac{P(B)}{P(B)} = 1\]
        \item $P(A \cap B) \leq P(B)$ since $A \cap B \subset B$, therefore $Q_B(A) \in [0,1]$.
    \end{enumerate}
\end{proof}
\textbf{Remark.} We have 
\[P(A^c|B) = 1 - P(A|B)\]
but we this does not apply for $B^c$,
\[P(A|B^c) \neq 1 - P(A|B)\]
\begin{theorem}[Law of Total Probability]
    Let $\{E_i\}$ be a sequence of disjoint events such that 
    \[\Omega = \bigcup_{i=1}^\infty E_i\]
    Then 
    \[P(A) = \sum_{i=1}^\infty P(A | E_i)P(E_i)\]
\end{theorem}
\begin{proof}
    \begin{align*}
        P(A) &= P(A \cap \Omega)\\
        &= P\left(A \cap \bigcup_{i=1}^\infty E_i\right)\\
        &= P\left(\bigcup_{i=1}^\infty A \cap E_i\right)\\
        &= \sum_{i=1}^\infty P(A \cap E_i)\\
        &= \sum_{i=1}^\infty P(A | E_i)P(E_i)
    \end{align*}
\end{proof}
\begin{theorem}[Bayes' Theorem]
    Let $\{E_i\}$ be a sequence of disjoint events such that 
    \[\Omega = \bigcup_{i=1}^\infty E_i\]
    Then 
    \[P(E_i | A) = \frac{P(A|E_i)P(E_i)}{P(A)} = \frac{P(A | E_i)P(E_i)}{\sum_{i=1}^\infty P(A | E_i)P(E_i)}\]
\end{theorem}
\begin{proof}
    
\end{proof}
\textbf{Example.} Roll a die and flip a coin the number of times that appears on the die. Let $X$ denote the random variable of the number of heads observed. What is the distrbiution of $X$? If we observed 3 heads, what is the probability that the die is 4?\\[2ex]
\textbf{Solution.} Let $Y$ denote the number on the die, then 
\begin{align*}
    P(X = k) &= \sum_{i=1}^{6} P(X = k | Y = i)P(Y = i)\\
    &= \sum_{i=1}{i \choose k} \left(\frac{1}{2}\right)^k \left(\frac{1}{2}\right)^{i-k} \frac{1}{6}\\
    &= \frac{1}{6}\sum_{i=1}^6 {i \choose k} \left(\frac{1}{2}\right)^i
\end{align*}
Then, we want to find $P(Y = 4 | X = 3)$, using Bayes' theorem
\begin{align*}
    P(Y = 4 | X = 3) &= \frac{P(X=3|Y=4)P(Y=4)}{P(X=3)}\\
    &= \frac{{4 \choose 3}\left(\frac{1}{2}\right)^k\frac{1}{2}\frac{1}{6}}{\frac{1}{6}\sum_{i=3}^6{i \choose 3}\left(\frac{1}{2}\right)^i}\\
    &= \frac{4}{16}
\end{align*}
\textbf{Example.} An urn contains $m+n$ chips of which $M$ are white and the rest are black. A chip is drawn at random without observing its color, then another chip is drawn. What is the probability that the second chip is white?\\[2ex]
\textbf{Solution.}
Let $E_1$ be the event that the first chip is white and $E_2$ be the event that the first chip is black. Also let $A$ be the event the second chip is black. We have 
\[P(E_1) = \frac{m}{m+n}\]
and 
\[P(A|E_1) = \frac{m-1}{m+n-1}, \ P(A|E_2) = \frac{m}{m+n-1}\]
Then, from the law of total probability, 
\begin{align*}
    P(A) &= P(A|E_1)P(E_1)+ P(A|E_2)P(E_2)\\
    &= \left(\frac{m-1}{m+n-1}\right)\left(\frac{m}{m+n}\right) + \left(\frac{m}{m+n-1}\right)\left(\frac{n}{m+n}\right) \\
    &= \frac{m}{m+n}
\end{align*}

\chapter{Expecation, Moments, Characteristic Functions and Functions of Random Variables.}

Let $X$ be a random variable with c.d.f $F$ such that 
\[\int_{-\infty}^\infty |U(x)|dF(x) < \infty\]
Then we can define 
\[E(U(X)) = \int_{-\infty}^\infty U(x)dF(x)\]
We denote the mean by $\mu = E(X)$, the $k$th moment by $\mu_k = E(X^k)$, and the variance by 
\[\sigma^2 = E[(X - E(X)^2)] = E(X^2) - E(X)^2\]  
with moment generating function
\[M(t) = E(e^{tX})\]

\begin{definition}[Characteristic Functions]
    Let $X$ be a random variable. The characteristic function of $X$ is defined by 
    \[\phi(t) = E(\exp(itX))\]
    where $i = \sqrt{-1}$. If $X$ is discrete then 
    \[g(s) = E(s^X) = \sum_k s^k P(X = k)\]
    is called the generating function.
\end{definition}
\noindent
\textbf{Example.} Show that 
\begin{enumerate}[label=(\roman*)]
    \item $f(x) = \frac{1}{\pi(1+x^2)}I(x \in \real)$ is a p.d.f.
    \item $E(X^k)$ does not exist if $k \geq 1$. 
\end{enumerate}
\begin{proof}
    It's clear that 
    \[\int_{-\infty}^\infty \frac{1}{\pi(1 + x^2)}dx = \left[\frac{\tan^{-1}(x)}{\pi}\right]^\infty_{-\infty} = 1\]
    Then, 
    \begin{align*}
        E(|X|^k) &= \int_{-\infty}^\infty \frac{|x|^k}{\pi(1+x^2)}dx\\
        &= \int_{0}^\infty \frac{x^k}{\pi(1+x^2)}dx + \int_{0}^\infty \frac{x^k}{\pi(1+x^2)}dx\\
        &= 2\int_{0}^\infty \frac{x^k}{\pi(1+x^2)}dx\\
        &\geq \frac{2}{\pi}\int_{1}^\infty \frac{x^k}{1+x^2}dx\\
        &\geq \frac{2}{\pi}\int_{1}^\infty \frac{x^k}{x^2}dx\\
    \end{align*}
    This integral diverges to infinity when $k > 1$, therefore $E(|X|^k)$ does not exist. 
\end{proof}
\textbf{Example.} Let $Z$ be a random variable with p.d.f 
\[f(z) = \frac{1}{\sqrt{2\pi}}e^{-z^2/2}\]
Find the c.d.f fpr $X = \sigma Z + \mu$ for $\sigma > 0$ and $\mu \in \real$. \\[2ex]
\textbf{Solution.} Note that $f(x) \geq 0$. Let 
\[I = \int_{-\infty}^{\infty} e^{-x^2/2}dx\]
Then, 
\[I^2 = \int_{-\infty}^\infty\int_{-\infty}^\infty \exp(-(x^2+y^2)/2)dxdy\]
Using polar coordinates, set $x = r\cos \theta$, and $y = r \sin \theta$, 
\[I^2 = \int_{0}^\infty \int_0^{2\pi} \exp(-r^2/2)rdrd\theta = 2\pi\]
Thus $I = \sqrt{2\pi}$. Now we can calculate 
\[F(x) = P(\sigma Z + \mu \leq X) = P\left(Z \leq \frac{x - \mu}{\sigma}\right) = \int_{-\infty}^{\frac{x-\mu}{\sigma}}f(z)dz\]
Then taking the derivative, we have
\[\frac{dF(x)}{dx} = \frac{1}{\sigma\sqrt{2\pi}}\exp\left(\frac{-(x-\mu)^2}{\sigma^2}\right)\] 
\subsection{Properties of Characteristic Functions}
The moment generating functions of a random variable may not exist. For example, it can be shown that the moment generating function for the Cauchy distribution may not exist 
\[\int_{-\infty}^{\infty}\frac{\exp(\theta x)}{\pi(1+x^2)}dx = \infty\]
However characteristic functions always exist.
\begin{theorem}
    If $X$ is a random variable, then 
    \begin{enumerate}[label=(\roman*)]
        \item $\phi(\theta) = E(\exp(i\theta X))$ always exists, with $\phi(0) = 1$ and $|\phi(\theta)| \leq 1$.
        \item $\overline{\phi(\theta)} = \phi(\theta)$ where $\overline{\phi(\theta)}$ is the complex conjugate.
        \item If $X$ is symmetric, then $\phi(X) \in \real$.
        \item $\phi(\theta)$ has linearity, i.e
        \[\phi_{aX + B}(\theta) = E(\exp(i\theta(aX+b))) = \phi_X(i\theta a)\exp(i\theta b)\]
        \item The characteristic function for any random variable $X$ is uniformly continuous.
        \item There is a 1 to 1 correspondence between a c.d.f of a random variable and its characteristic function, (Uniquness Theorem, without proof).
        \item $X_1$ and $X_2$ are independent if and only if 
        \[\phi(\theta_1,\theta_2) = E(\exp(i\theta_1X_1 + i\theta_2X_2)) = \phi_{X_1}(\theta_1)\phi_{X_2}(\theta_2)\]
    \end{enumerate}
\end{theorem}
\begin{proof}
    \begin{enumerate}[label=(\roman*)]
        \item Set $U = \cos(\theta)$, and $V = \sin(\theta)$, then using 
        \[e^{i\theta X} = \cos(\theta X) + i\sin(\theta X)\]
        We have 
        \[|E(\exp(i\theta X))|^2 = |E(U + iV)|^2 = E(U)^2 + E(V)^2 \leq E(U^2) + E(V^2) = 1\]
        \item Using the linearity of expecations, 
        \[\overline{E(\cos(\theta X)) + iE(\sin(\theta X))} = E(\cos(\theta X)) - iE(\sin(\theta X)) = \phi(-\theta)\]
        \item This follows from $X \eqd -X$.
        \item This follows from linearity of expecations,
        \[E(\exp(i\theta(aX + b))) = \exp(i\theta b)E(\exp(i\theta a X))\]
    \end{enumerate}
\end{proof}

    \begin{theorem}
        Let $X$ be a random variable with c.d.f $F$ such that $E(X)$ exists. Then 
        \[E(X) = \int_0^\infty (1- F(x))dx - \int_0^\infty F(-x)dx\]
    \end{theorem}
    \begin{proof}
        Assume $P(X \geq 0) = 1$, then 
        \begin{align*}
            E(X) &= \int_0^\infty xdF(x)\\
            &= \int_0^\infty\int_0^x dydF(x)\\
            &= \int_0^\infty\int_y^\infty dF(x)dy\\
            &= \int_0^\infty (1-F(y))dy\\
        \end{align*}
        In general, $X = X^+ - X^-$ where 
        \[X^+ = \max(0,X) = \frac{X + |X|}{2}\]
        \[X^- = \max(0,-X) = \frac{-X + |X|}{2}\]
        Threfore, $E(X) = E(X^+) - E(X^-)$. So we can write 
        \[E(X) = \int_0^\infty P(\max(0,X) > x)dx - \int_0^\infty P(\max(0,-X) > x)dx\]
        With 
        \begin{align*}
            P(\max(0,X) > x) &= P\left(\frac{X+|X|}{2} > x\right)\\
            &= P(|X| > 2x - X)\\
            &= P(X > 2x - X \text{ or } X < -2x + X) = P(X > x)\\
            P(X^- > x) &= P(|X| > 2x + X)\\
            &= P(X < -x)\\
        \end{align*}
    \end{proof}
    \noindent
    \textbf{Example.} Let $r \geq 0$, and $X \geq 0$, then 
    \begin{align*}
        E(X^r) &= \int_0^\infty P(X^r > x)dx\\
        &= \int_0^\infty P(X > x^{1/r})dx\\
        &= \int_0^\infty P(X > u)ru^{r-1}du\tag{$u = x^{1/r}$}   
    \end{align*}
    Therefore 
    \[E(X^r) = \int_0^\infty P(X^r > u)ru^{r-1}\]
    \textbf{Example.} 
    \begin{align*}
        E(|X|) &= \int_0^\infty P(|X| > x)dx\\
        &= \int_0^\infty (1 - P(|X| \leq x))dx\\
        &= \int_0^\infty (1 - P( - x \leq X \leq x))dx\\
        &= \int_0^\infty 1 - (F(x) - F(-x))dx\\
    \end{align*}
    \section{Distribution of Functions of Random Variables}
    Our goal is to find a c.d.f or p.d.f for a function $Y = U(X)$ for a random variable $X$.\\[2ex]
    \textbf{Example.} Let $X$ be a random variable with $f(x) = 2xI(0 \leq x \leq 1)$. Let $Y = X^2$. Find the p.d.f for $Y$.\\[2ex]
    \textbf{Solution.} We can compute the c.d.f directly 
    \begin{align*}
        G(y) = P(Y \leq y) &= P(X^2 \leq y)\\
        &= P(-\sqrt{y} \leq X \leq \sqrt{y})\\
        &= P(0 \leq X \leq \sqrt{y})\tag{Since $x \in [0,1]$}\\
        &= \int_0^{\sqrt{y}}2xdx\\
        &= \left[x^2\right]^{\sqrt{y}}_0 = y
    \end{align*}
    Therefore $G(y) = y$, and $g(y) = G'(y) = 1 I(0 \leq y \leq 1)$. This means $X^2 \sim \Unif(0,1)$. \\[2ex]
    \textbf{Example.} Let $X \sim f(x) = e^{-x}I(x > 0)$. Find the p.d.f for $Y = (\ln X)^2$.\\[2ex]
    \textbf{Solution.} Similarly to the previous example,
    \begin{align*}
        G(y) = P((\ln X)^2 \leq y) &= P(|\ln X| \leq \sqrt{y})\\
        &= P(-\sqrt{y} \leq \ln X \leq \sqrt{y})\\
        &= P(e^{-\sqrt{y}} \leq X \leq e^{\sqrt{y}})\\
        &= P(e^{-\sqrt{y}} < X \leq e^{\sqrt{y}})\\
        &= \int_{-\sqrt{y}}^{e^{\sqrt{y}}}e^{-x}dx = e^{\sqrt{y}} - e^{-\sqrt{y}}
    \end{align*} 
    Then 
    \[g(y) = G'(y) = \frac{1}{\sqrt{y}}e^{\sqrt{y}}I(0 < y < \infty)\]
    \textbf{Note.} The following formula will be useful for the next examples 
    \[\frac{\Gamma(\alpha)}{\beta^\alpha} = \int_0^\infty x^{\alpha - 1}\exp(-\beta x)dx\]
    \textbf{Example.} Let $X_1, X_2 \iid f(x) = \frac{1}{\Gamma(\alpha_i)}e^{-x}x^{\alpha_i - 1}I(x > 0)$. . Define $U = \frac{X_1}{X_1 + X_2}$. Find p.d.f for $U$. \\[1ex]
    \textbf{Solution.} Note that $0 \leq U \leq 1$ since $X_1 + X_2 \geq X_1$, so $U: [0,\infty) \times [0,\infty) \mapsto [0,1]$. We start with the c.d.f of $U$. We need to have the joint distribution for $X_1$ and $X_2$, so 
    \[f(x_1,x_2) = \frac{1}{\Gamma(\alpha_1)\Gamma(\alpha_2)}e^{-x_1}e^{-x_2}x_1^{\alpha_1-1}x_2^{\alpha_2-1}I(x_1, x_2 > 0)\]
    Now 
    \begin{align*}
        G(u) &= P\left(\frac{X_1}{X_1 + X_2} \leq u\right)\\
        &= P(X_1 \leq u(X_1 + X_2))\\
        &= P(X_1 - uX_1 \leq uX_2)\\
        &= P\left(X_1 \leq \frac{uX^2}{1-u}\right)\\
        &= \int_0^\infty\int_0^{\frac{x_2u}{1-u}} \frac{e^{-x_1 - x_2}}{\Gamma(\alpha_1)\Gamma(\alpha_2)}x_1^{\alpha_1-1}x_2^{\alpha_2 - 1}dx_1dx_2\\
    \end{align*}
    Differentiating this function we get 
    \begin{align*}
        g(u) = G'(u) &= \frac{\Gamma(\alpha_1 + \alpha_2)}{\Gamma(\alpha_1)\Gamma(\alpha_2)}u^{\alpha_1 - 1}(1-u)^{\alpha_2 - 1}
    \end{align*}
    \textbf{Note.} From the beta distribution with parameters $\alpha_1$, $\alpha_2$ we get the useful formula 
    \[\int_0^1 \frac{\Gamma(\alpha_1 + \alpha_2)}{\Gamma(\alpha_1)\Gamma(\alpha_2)}u^{\alpha_1-1}(1-u)^{\alpha_2 - 1} du = 1\]
    \[\implies \int_0^1 u^{\alpha_1-1}(1-u)^{\alpha_2 - 1} du = \frac{\Gamma(\alpha_1 + \alpha_2)}{\Gamma(\alpha_1)\Gamma(\alpha_2)}\]
    Another way to solve the previous example is to introduce another variable $V = X + Y$. The joint distribution for $X$ and $Y$ 
    \[f(x,y) = \frac{1}{\Gamma(\alpha_1)\Gamma(\alpha_2)}e^{-(x+y)}x^{\alpha_1 - 1}y^{\alpha_2-1}\]
    Then we can solve for $X$ and $Y$ in terms of $U$ and $V$ and we get $X = UV$, $Y = V(1 - U)$. Then the Jacobian for this transformation is 
    \[J = \left|\begin{matrix}
        \frac{\partial X}{\partial u} & \frac{\partial X}{\partial v}\\
        \frac{\partial Y}{\partial u} & \frac{\partial y}{\partial v}\\
    \end{matrix}\right| \left|\begin{matrix}
        v & u\\
        -v & 1 - u
    \end{matrix}\right| = v\]
    Now our joint p.d.f for $U,V$ is  
    \begin{align*}
        g(u,v) &= f(x(u,v), y(u,v))|J| = f(uv, v(1-u))|J|\\
        &= \frac{1}{\Gamma(\alpha_1)\Gamma(\alpha_2)}e^{-v}(uv)^{\alpha_1 - 1}(v(1-u))^{\alpha_2-1}vI(0 < v < 1, u > 0)
    \end{align*}
    We don't want the joint p.d.f for $U$ and $V$, we want it for $U$ so we can integrate over $V$ 
    \begin{align*}
        g(u) = \int_0^\infty g(u,v)dv &= \int_0^\infty \frac{1}{\Gamma(\alpha_1)\Gamma(\alpha_2)}e^{-v}(uv)^{\alpha_1 - 1}(v(1-u))^{\alpha_2-1}vdv\\
        &= \frac{1}{\Gamma(\alpha_1)\Gamma(\alpha_2)}u^{\alpha_1 - 1}(1-u)^{\alpha_2-1}\int_0^\infty e^{-v}v^{\alpha_1 + \alpha_2 + 1}dv
    \end{align*}
    Notice that the integral is $\Gamma(\alpha_1 + \alpha_2)$ so we get 
    \[g(u) = \frac{\Gamma(\alpha_1 + \alpha_2)}{\Gamma(\alpha_1)\Gamma(\alpha_2)}u^{\alpha_1 - 1}(1-u)^{\alpha_2-1} \]

    \textbf{Example.} Let $X \sim \Unif(0,1)$. Find the p.d.f for 
    \begin{enumerate}[label=(\roman*)]
        \item $Y = a+(b-a)X$ where $a < b$
        \item $W = \tan\left(\frac{\pi(2X-1)}{2}\right)$
    \end{enumerate}
    \textbf{Solution.} 
    \begin{enumerate}[label=(\roman*)]
        \item Let $G$ denote the c.d.f for $Y$, then 
        \begin{align*}
            G(y) = P(Y \leq y) &= P(a + (b-a)X \leq y)\\
            &= P((b-a)X \leq y - a)\\
            &= P\left(X \leq \frac{y-a}{b-a}\right)\\
            &= \frac{y-a}{b-a}   
        \end{align*}
        Then the p.d.f is 
        \[g(y) = G'(y) = \frac{1}{b-a}I(a \leq y \leq b)\]
        Therefore, $Y \sim \Unif(a,b)$.
        
        \item Let $H$ denote the c.d.f for $W$, then 
        \begin{align*}
            H(w) = P(W \leq w) &= P\left(\tan\frac{\pi(2X-1)}{2} \leq 2\right)\\
            &= P\left(\frac{\pi(2X-1)}{2} \leq \arctan(w)\right)\\
            &= P\left(X \leq \frac{2\arctan(w) + 1}{2\pi}\right)\\
            &= \frac{2\arctan(w) + 1}{2\pi}
        \end{align*}
    Then the p.d.f is 
    \[h(w) = H'(w) = \frac{1}{\pi(1+w^2)}I(w \in \real)\]
    Thus $W$ has a Cauchy distribution.\\[2ex]

    \textbf{Example.} Let $X$ and $Y$ be two independent random variables such that 
    \[X \sim f(x) = \frac{1}{\Gamma(\alpha)}e^{-x}x^{\alpha - 1}=\]
    \[Y \sim h(y) = \frac{1}{\Gamma(\beta)} e^{-y}y^{\beta -1}\]
    Find the p.d.f for the random variable $\frac{X}{X + Y}$.\\[1ex]
    \textbf{Solution.} The c.d.f is given as 
    \begin{align*}
        G(u) = P\left(\frac{X}{X+Y} \leq u\right) &= P\left(X \leq \frac{yu}{1-u}\right)\\
        &= \frac{1}{\Gamma(\alpha)\Gamma(\beta)}\int_0^\infty\int_0^{\frac{yu}{1-u}} \exp(-x)\exp(-y)x^{\alpha -1}y^{\beta - 1}dxdy\\
    \end{align*}

    \end{enumerate}


\end{document}