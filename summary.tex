
\documentclass{article}
\usepackage[landscape]{geometry}
\usepackage{url}
\usepackage{multicol}
\usepackage{amsmath}
\usepackage{esint}
\usepackage{amsfonts}
\usepackage{tikz}
\usetikzlibrary{decorations.pathmorphing}
\usepackage{amsmath,amssymb}
\usepackage{listings}
\usepackage{colortbl}
\usepackage{xcolor}
\usepackage{mathtools}
\usepackage{amsmath,amssymb}
\usepackage{enumitem}
\usepackage{environ}
\makeatletter

\newcommand*\bigcdot{\mathpalette\bigcdot@{.5}}
\newcommand*\bigcdot@[2]{\mathbin{\vcenter{\hbox{\scalebox{#2}{$\m@th#1\bullet$}}}}}
\makeatother

\title{CSI 2132 Midterm Cheat Sheet}
\usepackage[brazilian]{babel}
\usepackage[utf8]{inputenc}

\advance\topmargin-.8in
\advance\textheight3in
\advance\textwidth3in
\advance\oddsidemargin-1.5in
\advance\evensidemargin-1.5in
\parindent0pt
\parskip2pt
\newcommand{\hr}{\centerline{\rule{3.5in}{1pt}}}
%\colorbox[HTML]{e4e4e4}{\makebox[\textwidth-2\fboxsep][l]{texto}


\definecolor{blue}{HTML}{A7BED3}
\definecolor{brown}{HTML}{DAB894}
\definecolor{pink}{HTML}{FFCAAF}


\newtheorem{theorem}{Theorem}[section]
\newtheorem{definition}{Definition}[section]
\newtheorem{fact}{Fact}[section]
\newtheorem{prop}{Proposition}[section]
\newtheorem{corollary}{Corollary}[section]





\tikzset{header/.style={path picture={
\fill[green, even odd rule, rounded corners]
(path picture bounding box.south west) rectangle (path picture bounding box.north east) 
([shift={( 2pt, 4pt)}] path picture bounding box.south west) -- 
([shift={( 2pt,-2pt)}] path picture bounding box.north west) -- 
([shift={(-2pt,-4pt)}] path picture bounding box.north east) -- 
([shift={(-6pt, 6pt)}] path picture bounding box.south east) -- cycle;
},
label={[anchor=west, fill=green]north west:\textbf{#1:}},
}} 

\tikzstyle{mybox} = [draw=black, fill=white, very thick,
    rectangle, rounded corners, inner sep=10pt, inner ysep=10pt]
\tikzstyle{fancytitle} =[fill=black, text=white, rounded corners, font=\bfseries]


\tikzstyle{bluebox} = [draw=blue, fill=white, very thick,
    rectangle, rounded corners, inner sep=10pt, inner ysep=10pt]
\tikzstyle{bluetitle} =[fill=blue, inner sep=4pt, text=white, font=\small]


\tikzstyle{brownbox} = [draw=brown, fill=white, very thick,
    rectangle, rounded corners, inner sep=10pt, inner ysep=10pt]
\tikzstyle{browntitle} =[fill=brown, inner sep=4pt, text=white, font=\small]

\tikzstyle{pinkbox} = [draw=pink, fill=white, very thick,
    rectangle, rounded corners, inner sep=10pt, inner ysep=10pt]
\tikzstyle{pinktitle} =[fill=pink, inner sep=4pt, text=white, font=\small]

\tikzstyle{redbox} = [draw=red!35, fill=white, very thick,
    rectangle, rounded corners, inner sep=10pt, inner ysep=10pt]
\tikzstyle{redtitle} =[fill=red!35, inner sep=4pt, text=white, font=\small]


\NewEnviron{brownbox}[1]{
    \begin{tikzpicture}
    \node[brownbox](box){%
    \begin{minipage}{0.9\textwidth}
    \BODY
    \end{minipage}};
    \node[browntitle, right=10pt] at (box.north west) {#1};
    \end{tikzpicture}
}

 \NewEnviron{redbox}[1]{
    \begin{tikzpicture}
    \node[redbox](box){%
    \begin{minipage}{0.9\textwidth}
    \BODY
    \end{minipage}};
    \node[redtitle, right=10pt] at (box.north west) {#1};
    \end{tikzpicture}
}
   
    

\NewEnviron{bluebox}[1]{%
\begin{tikzpicture}
    \node[bluebox](box){%
        \begin{minipage}{0.9\textwidth}
            \BODY
        \end{minipage}
    };
    
\node[bluetitle, right=10pt] at (box.north west) {#1};
\end{tikzpicture}
}

\NewEnviron{pinkbox}[1]{%
\begin{tikzpicture}
    \node[pinkbox](box){%
        \begin{minipage}{0.9\textwidth}
            \BODY
        \end{minipage}
    };
    
\node[pinktitle, right=10pt] at (box.north west) {#1};
\end{tikzpicture}
}



\NewEnviron{blackbox}[1]{%
\begin{tikzpicture}
    \node[mybox](box){%
        \begin{minipage}{0.3\textwidth}
        \raggedright
        \small{
            \BODY
        }
        \end{minipage}
    };
    
\node[fancytitle, right=10pt] at (box.north west) {#1};
\end{tikzpicture}
}


\begin{document}

\begin{center}{\large{\textbf{MAT 3172 Summary Sheet}}}\\
\end{center}




\begin{multicols*}{3}
\begin{blackbox}{Indicator Functions}
    \textbf{Definition.} Let $A \subset \Omega$. The indicator function of $A$ is defined as 
    \[I(x \in A) = I_A(x) = \begin{cases}
        1 & x \in A \\
        0 & x \notin A
    \end{cases}\]
    \begin{bluebox}{Properties of Indicator Functions}
        \begin{itemize}
            \item $I_{A \cup B} = \max(I_A, I_B)$
            \item $I_{A \cap B} = I_A \cdot I_B$
            \item $I_{A \Delta B} = I_A + I_B \pmod{2}$
            \item $A \subset B$ if and only if $I_A \leq I_B$
            \item $I_{\cup_i A_i} \leq \sum_i I_{A_i}$
        \end{itemize}
    \end{bluebox}\\[-2ex]
\end{blackbox}
\begin{blackbox}{Set Theoretic Limits}
    \textbf{Definition.} Let $\{A_n\}$ be a sequence of events,
    \[\liminf A_n = \bigcup_{n=1}^\infty \bigcap_{m=n}^\infty A_m\]
    \[\limsup A_n = \bigcap_{n=1}^\infty \bigcup_{m=n}^\infty A_m\]
    \begin{redbox}{Lemma}
        \[\limsup A_n = \left\{
            \omega: \sum_{i=1}^\infty I_{A_i}(\omega) = \infty
        \right\}\]
        \[\liminf A_n = \left\{
            \omega: \sum_{i=1}^\infty I_{A_i^c}(\omega) < \infty
        \right\}\]
    \end{redbox}
    To summarize, with the limit superior we have infinitely many cases where
    $I_{A_i} (\omega) = 1$. For the limit inferior, it means that $\omega$ is in all but finitely many of
    the $A_i$’s.
    \begin{brownbox}{Lemma}
        Let ${A_n}$ be a sequence of events. Then
        \begin{enumerate}
            \item If $A_n \subset A_{n+1}$ for any integer $n$, then 
            \[\lim A_n = \bigcup_{n=1}^\infty A_n\]
            \item If $A_{n+1} \subset A_n$ for any integer $n$, then 
            \[\lim A_n \bigcap_{n=1}^\infty A_n\]
        \end{enumerate}
    \end{brownbox}\\[-2ex]
\end{blackbox}
\begin{blackbox}{More on Set Theoretic Limits}
    \begin{redbox}{Lemma}
        Let $\{A_n\}$ be a sequence of events, then 
        \begin{enumerate}
            \item If $A_n \subset A_{n+1}$ for any integer $n$, then \\[-2ex]
            \[\lim A_n = \bigcup_{n=1}^\infty A_n\]
            \item If $A_{n+1} \subset A_n$ for any integer $n$, then \\[-2ex]
            \[\lim A_n = \bigcap_{n=1}^\infty A_n\]  
        \end{enumerate}
    \end{redbox}\\[-2ex]
\end{blackbox}
\begin{blackbox}{Algebras}
    \textbf{Definition.} An \emph{algebra} (also called a field) $\mathcal{A}$ is a class of subsets of $\Omega$ (called events) that satisfies
    \begin{itemize}
        \item $\Omega \subset \mathcal{A}$ 
        \item If $A \in \mathcal{A}$, then $A^c \in \mathcal{A}$
        \item If $A,B \in \mathcal{A}$, then $A \cup B \in \mathcal{A}$
    \end{itemize}
    \textbf{Definition.} A \emph{$\sigma$-algebra} (also called a \emph{$\sigma$-field}) is an algebra that is closed under countable union.  
\end{blackbox}
\begin{blackbox}{Probability Measures}
    \textbf{Definition.} Let $\Omega$ be a sample space and $\mathcal{F}$ be a $\sigma$-field on $\Omega$. A probability measure $P$ is defined on $\mathcal{F}$ such that 
    \begin{enumerate}[label=(\roman*)]
        \item $P(\Omega) = 1$
        \item If $A_1, A_2, \ldots \in \mathcal{F}$ are pairwise disjoint, then\\[-2ex]
        \[P\left(\bigcup_{i=1}^\infty A_i\right) = \sum_{i=1}^\infty P(A_i)\]
    \end{enumerate}
    \begin{bluebox}{Properties of Probability Measures}
        \begin{enumerate}[label=(\roman*)]
            \item $P(\Omega) = 1 = P(\Omega \cup \emptyset) = P(\Omega) + P(\emptyset) \implies P(\emptyset) = 0$
            \item $(A\setminus B) \cup (A \cap B) = A$ and $(A\setminus B) \cap (A \cap B) = \emptyset$, we have\\[-2ex]
            \[P(A \setminus B) = P(A) - P(A \cap B)\]
            \item Similarly, $(A \setminus B) \cup B = A \cup B$ and $(A \setminus B) \cap B = \emptyset$, which implies\\[-2ex] 
            \[P(A \cup B) = P(A) + P(B) - P(A \cap B)\]
            \item If $A \subset B$ then $A \cup (B \setminus A) = B$. Therefore, \\[-1ex]
            \[P(A) + P(B \setminus A) = P(B)\]
            and furthermore,
            $P(A) \leq P(B)$
        \end{enumerate}
    \end{bluebox}\\[-2ex]
\end{blackbox}
\begin{blackbox}{Expectations}
    \textbf{Definition.} Let $X: \Omega \rightarrow \mathbb{R}$, an expecation $E$ is an operator with the following properties, 
    \begin{enumerate}[label=(\roman*)]
        \item If $X \geq 0$, then $E(X) \geq 0$
        \item If $c \in \mathbb{R}$ is a constant, then $E(cX) = cE(X)$
        \item $E(X_1 + X_2) = E(X_1) + E(X_2)$
        \item $E(1) = 1$
        \item If $X_n(\omega)$ is monotonically increasing and $X_n(\omega) \rightarrow X(\omega)$, then 
        \[\lim_{n\rightarrow\infty} E(X_n) = E(X)\]
    \end{enumerate}
    \begin{bluebox}{Finding Probabilities Using Expecations}
        \textbf{Definition.} For any event $A$, define 
        \[P(A) = E(I_A(\omega))\]
        \raggedright
        For simplicity we write $P(A) = E(I_A)$.
        \begin{redbox}{Properties}
            \begin{enumerate}[label=(\roman*)]
                \item $E\left(\sum\limits_{i=1}^n c_iX_i\right) = \sum\limits_{i=1}^n c_iE(X_i)$
                \item If $X \leq Y \leq Z$, then $E(X) \leq E(Y) \leq E(Z)$
                \item If $\{A_i\}$ is a sequence of events, then 
                \[P\left(\bigcup_{i=1}^\infty A_i\right) \leq \sum_{i=1}^\infty P(A_i)\]
            \end{enumerate}
        \end{redbox}
    \end{bluebox}
    \begin{brownbox}{Fatou's Lemma}
        If $\{A_n\}$ is a family of events, then 
        \raggedright
        \begin{enumerate}
            \item $$P(\liminf A_n) \leq \liminf P(A_n) $$
            \[\leq \limsup P(A_n) \leq P(\limsup A_n)\]
            \item If $\lim A_n = A$, then $\lim P(A_n) = P(A)$
        \end{enumerate}
    \end{brownbox}
    \begin{pinkbox}{Lemma}
        If $A_n \subset A_{n+1}$ for any $n$, then $\lim P(A_n) = P(A)$. If $A_{n+1} \subset A_n$ for any $n$, then $\lim P(A_n) = P(A)$ where $\lim A_n = A$. 
    \end{pinkbox}
\end{blackbox}
\end{multicols*}
\end{document}
